Consider $\cE^{\odet}$, a deterministic eVA. Let $(S_1, i_1)\cdot \ldots \cdot (S_m, i_m)$ be an output sequence and $d = a_0\ldots a_n$ a document. Assume that $(S_1, i_1)\cdot \ldots \cdot (S_n, i_n)$ is in $\semseq{\cE}_d$ (the case where it is not is trivial). Under this assumption, there must exist at least one accepting run $\rho$ of $\cE^{\odet}$ over $d$:
$$
\rho \ = \ q_0 \ \longtrans{b_0 / R_0} \ q_1 \ \longtrans{b_1 / R_1} \ q_2 \ \longtrans{b_2 / R_2} \ \ldots \ \ q_{n} \longtrans{b_{m} / R_n} \ q_{n+1}
$$
such that $\oout(\rho) = (S_1, i_1)\cdot \cdots \cdot (S_m, i_m)$. Now, suppose there is another accepting run $\rho'$ of $\cE^{\odet}$:
$$
\rho' \ = \ p_0 \ \longtrans{b_0 / T_0} \ q_1 \ \longtrans{b_1 / T_1} \ p_2 \ \longtrans{b_2 / T_2} \ \ldots \ \ p_{\ell} \longtrans{b_n / T_n} \ p_{n+1}
$$
over $d$, where $\rho' \neq \rho$ but $\oout(\rho') = \oout(\rho)$. First, observe that
\begin{equation} \label{eq:outseq}
\oout(R_0, 0)\cdot \ldots \cdot \oout(R_n, n) = \oout(T_0, 0)\cdot \ldots \cdot \oout(T_n, n) = (S_1, i_1)\cdot \ldots \cdot (S_m, i_m).
\end{equation}
This implies that $R_j = T_j$ for all $0 \leq j \leq n$. If there were some $j$ where $R_j \neq T_j$, then $\oout(R_j, j) \neq \oout(T_j, j)$ would hold. This would mean that $R_j$ and $T_j$ cannot both be empty simultaneously, and thus, at least one of $(R_j, j)$ and $(T_j, j)$ would be present in the output sequence. However, this would contradict equation \eqref{eq:outseq} since the sequences in it are strictly increasing with respect to the positions in the document.

Next, considering that $\rho'$ and $\rho$ are distinct and $q_0 = p_0$, there must exist a smallest index $0 \leq j \leq m$ for which $q_j = p_j$ and $q_{j+1} \neq p_{j+1}$. This implies that both $(q_j, b_j, R_j, q_{j+1})$ and $(q_j, b_j, R_j, p_{j+1})$ are transitions in $\cE^\odet$. However, this is a contradiction because $\cE^{\odet}$ is deterministic, and hence it cannot have two different transitions from the same state with the same input and output. Therefore, we must conclude that $\rho' = \rho$, establishing the uniqueness of $\rho$.