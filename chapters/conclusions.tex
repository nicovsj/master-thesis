% !TeX root = ../Thesis.tex

In this thesis, we have meticulously analyzed the methods necessary to construct
an information extraction regex engine capable of supporting all-match
semantics, culminating in the development of the \texttt{REmatch} engine. A key
highlight is the use of enumeration algorithms as the fundamental framework,
notably the ECS data structure, which facilitates output-linear delay
enumeration. We have also implemented multiple optimizations to enhance the
engine's runtime efficiency and memory usage.

The \texttt{REmatch} engine proves to be a strong competitor against leading
regex engines, despite performing more complex tasks to ensure the extended
semantics not supported by other engines. The utility of all-match semantics, as
evidenced in our datasets, is particularly relevant in areas like DNA analysis
or linguistics, where overlapping matches are desirable in some contexts.

Future development plans must acknowledge that \texttt{REmatch} is not yet a
complete regex library for information extraction. Currently, the engine
requires more extensive testing and comprehensive documentation for it to be
able to participate in a in a production setting. In terms of performance
enhancements, several areas for optimization exist. For instance, the existing
\textit{all-match} algorithm does not feature parallel processing, a
modification that could significantly boost runtime efficiency. Additionally,
the potential expansion of the REQL query language to include more operators and
semantics is an exciting prospect. A notable improvement could be the relaxation
of constraints in REQL queries, allowing for captures under quantifier operators
(\texttt{*}, \texttt{+}, \texttt{?}, \texttt{\{n,m\}}). This adjustment would
allow a variable to correspond to multiple spans, providing increased
versatility in applications such as extracting multiple values from rows in CSV
files.
