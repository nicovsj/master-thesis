% !TeX spellcheck = en_US !TeX root = ../main/main.tex

The first step for the evaluation of a REQL query is the compilation and
rewriting into a logical plan, called a \emph{logical VA} (short for logical
variable-set automaton). This plan is essentially an automaton with variables
that is equally expressive as a REQL query. Furthermore a logical VA is suitable
for rewriting. Specifically, we perform an \emph{offset} transformation over the
logical VA that keeps the semantics of the query but improves the performance of
the evaluation algorithm. Next we explain these two components. %of \rematch.

\section{Logical VA}

A \emph{logical variable-set automata} (logical VA) is a non-deterministic
finite state automaton extended with captures variables.
% in a way analogous to REQL.
Formally, a logical VA $\cA$ is a tuple $(Q, \delta, q_0, q_f)$, where $Q$ is a
finite set of states, $q_0$ and $q_f$ are the initial and the final state, and
$\delta$ is a transition relation consisting of letter transitions $(q, C, q')$,
and variable transitions $(q, \Open{x}, q')$ or $(q, \Close{x}\,,q')$, where $q,
q' \in Q$, $C$ is a char class (e.g., a letter $\texttt{a}$, $\texttt{[w]}$ or
$\texttt{[\textasciicircum w]}$) and $x$ is a variable. The $\Open{x}$ and
$\Close{x}$ are special symbols to denote the opening or closing of a variable
$x$. In the following, we refer to $\Open{x}$ and $\Close{x}$ collectively as
\emph{variable markers}.

A configuration of a logical $\VA$ over a document $d$ is a tuple $(q, i)$ where
$q \in Q$ is the current state and \(i \in [0, |d|]\) is the current position in
$d$. A run $\rho$ over $d = a_0 a_1 \cdots a_{n-1}$ is a sequence:
$$
	\rho \ = \ (q_0, i_0) \ \trans{o_1} \ (q_1, i_1) \ \trans{o_2} \ \cdots \ \trans{o_m} \ (q_m, i_m)
$$
where $(q_j, o_{j+1}, q_{j+1}) \in \delta$,  $i_0, \ldots, i_m$ is an increasing
sequence, and $i_{j+1} = i_j +1$ if $o_{j+1}$ is a char class such that $a_{i_j}
\in \cset(o_{j+1})$   (i.e.,\ the automaton moves one position in the document
only when reading a letter) and $i_{j+1} = i_j$ otherwise. Furthermore, $\rho$
must satisfy that variables are opened and closed in a correct manner, namely,
each $x$ is closed at most once and only if it is opened previously. We say that
$\rho$ is \emph{accepting} if $q_m = q_f$, in which case we define the mapping
$\mu^{\rho}$ that maps $x$ to $[i_j, i_k\rangle$ if, and only if, $o_{i_j} =
\Open{x}$ and $o_{i_k} = \Close{x}$ in~$\rho$. Notice that we do not require
that $i_0 = 0$, nor $i_m=n$; namely, an accepting run can start or end at any
position in the document $d$, as long as it consumes a contiguous substring of
$d$. Finally, the semantics of $\cA$ over $d$, denoted by \(\semd{\cA}\) is
defined as the set of all $\mu^{\rho}$ where $\rho$ is an accepting run of $\cA$
over $d$.

\begin{example}\label{example:logicalVA} Consider the REQL query \texttt{e0} of
	Example~\ref{example:basic}. The following is a logical VA representing
	\texttt{e0}:
	\begin{center}
		\begin{tikzpicture}[->,>=stealth, semithick, auto, initial text= {}, initial distance= {3mm}, accepting distance= {4mm}, node distance=1.7cm, semithick]
			\tikzstyle{every state}=[draw=black,text=black,inner sep=0pt,
			minimum size=7mm]

			\begin{scope}
				\node[initial,state] (0) {$0$}; \node[state] (1) [right of=0]
				{$1$}; \node[state] (2) [right of=1] {$2$}; \node[state] (3)
				[right of=2] {$3$}; \node[state] (4) [right of=3] {$4$};
				\node[state] (5) [right of=4] {$5$}; \node[state,accepting] (6)
				[right of=5] {$6$};

				\draw[->] (0) edge node {$\Open{x}$} (1); \draw[->] (1) edge
				node {$t$} (2); \draw[->] (2) edge node {$h$} (3); \draw[->] (3)
				edge node {$a$} (4); \draw[->] (4) edge node {$t$} (5);
				\draw[->] (5) edge node {$\Close{x}$} (6);
			\end{scope}
		\end{tikzpicture}
	\end{center}
	In this figure, the states are $\{0, \ldots, 5\}$, where $0$ and $5$ are the
	initial and final state, respectively.  The edges between states are
	transitions, where the first and last edges are variable transitions, i.e.,
	they open and close $x$ with the variable markers $\Open{x}$ and
	$\Close{x}$, respectively, and the middle edges are letter transitions.
\end{example}
Example~\ref{example:logicalVA} shows how to compile a REQL query into a logical
VA. Using a construction similar to Thomson's~\citep{HopcroftU79}, we can covert
every REQL query into a logical VA, giving us a logical plan for the query.
\begin{proposition} \label{prop:logicalVA} For every REQL query $\texttt{e}$,
	one can build in linear time a logical VA $\cA$  such that
	$\sem{\texttt{e}}_d = \sem{\cA}_d$ for every document~$d$.
\end{proposition}
%\cristian{Notar que como esta actualmente el modelo, no podemos compilar los
%caracteres especiales de inicio y fin de documento. Tampoco están soportados
%por REmatch actualmente como dice Nico, así que debemos tomar una decisión.}
The proof for proposition \ref*{prop:logicalVA} is included in Appendix
\ref*{app:logicalVA}, with the details of the construction. Note that logical VA
is an extension of \emph{variable-set automata} (VA) from~\citet{FaginKRV15}.
The main difference between the two models is that logical VA uses char classes
in its letter transitions whereas VA uses individual letters. Moreover, a
logical VA can start a run at any position, whereas VA starts from the beginning
of the document. Although both models are equally expressive, we use logical VAs
as logical plans for compiling REQL formulas in practice.

\section{Offsets (optimization)}

In some cases, opening a variable can be postponed in order not to store the
information about runs that will not result in an output. To illustrate this,
consider again the expression \texttt{e0} and its logical VA of
Example~\ref{example:logicalVA}. Intuitively, our algorithm needs to store the
position information for the opening of a variable $x$ every time a \texttt{t}
would be read. If the document we are reading has the text \texttt{thasty}, this
run would then be extended for two more steps, although it will eventually be
abandoned, and not result in any outputs. In cases such as these, we can
actually postpone (offset) opening of the variable $x$ by transforming the
logical VA as follows: (i) first read the word \texttt{that}; (ii) now open a
variable marker $\Open{x}$, but remember that it was actually opened four
symbols before (i.e., it has an offset 4); (iii) proceed with the current run.
When reconstructing the output, we will start reading four symbols before the
position that is stored for $\Open{x}$. The transformation of the logical VA
from Example~\ref{example:logicalVA} would look as follows:
\begin{center}
	\begin{tikzpicture}[->,>=stealth, semithick, auto, initial text= {}, initial distance= {3mm}, accepting distance= {4mm}, node distance=1.7cm, semithick]
		\tikzstyle{every state}=[draw=black,text=black,inner sep=0pt, minimum
		size=7mm]

		\begin{scope}
			\node[initial,state] (0) {$0$}; \node[state] (1) [right of=0] {$1$};
			\node[state] (2) [right of=1] {$2$}; \node[state] (3) [right of=2]
			{$3$}; \node[state] (4) [right of=3] {$4$}; \node[state] (5) [right
			of=4] {$5$}; \node[state,accepting] (6) [right of=5] {$6$};

			\draw[->] (0) edge node {$t$} (1); \draw[->] (1) edge node {$h$}
			(2); \draw[->] (2) edge node {$a$} (3); \draw[->] (3) edge node
			{$t$} (4); \draw[->] (4) edge node {$\Open{x}^{-4}$} (5); \draw[->]
			(5) edge node {$\Close{x}$} (6);
		\end{scope}
	\end{tikzpicture}
\end{center}
The notation $\Open{x}^{-4}$ is used in order to signal that the variable $x$
was actually opened four positions before it was recorded in the run. %when
opened, the variable \texttt{x} actually stores characters starting four
positions before the guarded position.

Offsets are implemented in the rewriting module of \rematch\ after constructing
the first logical VA from a REQL query. When the query contains quantifiers
(\texttt{*}, \texttt{+}, \texttt{?} and \texttt{\{$n$,$m$\}} operators) or
alternations (\texttt{|} operator) special care must be taken for offsetting the
variables. More details are provided in Appendix \ref{app:offset}.
