% !TeX spellcheck = en_US !TeX root = ../main/main.tex

The first step for the evaluation of a REQL query is the compilation and
rewriting into a logical plan, called a \emph{logical VA} (short for logical
variable-set automaton). This plan is essentially an automaton with variables
that is equally expressive as a REQL query. Furthermore a logical VA is suitable
for rewriting. Specifically, we perform an \emph{offset} transformation over the
logical VA that keeps the semantics of the query but improves the performance of
the evaluation algorithm. Next we explain these two components. %of \rematch.

\section{Logical VA}

A \emph{logical variable-set automata} (logical VA) is a non-deterministic
finite state automaton extended with captures variables.
% in a way analogous to REQL.
Formally, a logical VA $\cA$ is a tuple $(Q, \delta, q_0, q_f)$, where $Q$ is a
finite set of states, $q_0$ and $q_f$ are the initial and the final state, and
$\delta$ is a transition relation consisting of letter transitions $(q, C, q')$,
and variable transitions $(q, \Open{x}, q')$ or $(q, \Close{x}\,,q')$, where $q,
q' \in Q$, $C$ is a char class (e.g., a letter $\texttt{a}$, $\texttt{[w]}$ or
$\texttt{[\textasciicircum w]}$) and $x$ is a variable. The $\Open{x}$ and
$\Close{x}$ are special symbols to denote the opening or closing of a variable
$x$. In the following, we refer to $\Open{x}$ and $\Close{x}$ collectively as
\emph{variable markers}.

A configuration of a logical $\VA$ over a document $d$ is a tuple $(q, i)$ where
$q \in Q$ is the current state and \(i \in [0, |d|]\) is the current position in
$d$. A run $\rho$ over $d = a_0 a_1 \cdots a_{n-1}$ is a sequence:
$$
	\rho \ = \ (q_0, i_0) \ \trans{o_1} \ (q_1, i_1) \ \trans{o_2} \ \cdots \ \trans{o_m} \ (q_m, i_m)
$$
where $(q_j, o_{j+1}, q_{j+1}) \in \delta$,  $i_0, \ldots, i_m$ is an increasing
sequence, and $i_{j+1} = i_j +1$ if $o_{j+1}$ is a char class such that $a_{i_j}
\in \cset(o_{j+1})$   (i.e.,\ the automaton moves one position in the document
only when reading a letter) and $i_{j+1} = i_j$ otherwise. Furthermore, $\rho$
must satisfy that variables are opened and closed in a correct manner, namely,
each $x$ is closed at most once and only if it is opened previously. We say that
$\rho$ is \emph{accepting} if $q_m = q_f$, in which case we define the mapping
$\mu^{\rho}$ that maps $x$ to $[i_j, i_k\rangle$ if, and only if, $o_{i_j} =
\Open{x}$ and $o_{i_k} = \Close{x}$ in~$\rho$. Notice that we do not require
that $i_0 = 0$, nor $i_m=n$; namely, an accepting run can start or end at any
position in the document $d$, as long as it consumes a contiguous substring of
$d$. Finally, the semantics of $\cA$ over $d$, denoted by \(\semd{\cA}\) is
defined as the set of all $\mu^{\rho}$ where $\rho$ is an accepting run of $\cA$
over $d$.

\begin{example}\label{example:logicalVA} Consider the REQL query \texttt{e0} of
	Example~\ref{example:basic}. The following is a logical VA representing
	\texttt{e0}:
	\begin{center}
		\begin{tikzpicture}[->,>=stealth, semithick, auto, initial text= {}, initial distance= {3mm}, accepting distance= {4mm}, node distance=1.7cm, semithick]
			\tikzstyle{every state}=[draw=black,text=black,inner sep=0pt,
			minimum size=7mm]

			\begin{scope}
				\node[initial,state] (0) {$0$}; \node[state] (1) [right of=0]
				{$1$}; \node[state] (2) [right of=1] {$2$}; \node[state] (3)
				[right of=2] {$3$}; \node[state] (4) [right of=3] {$4$};
				\node[state] (5) [right of=4] {$5$}; \node[state,accepting] (6)
				[right of=5] {$6$};

				\draw[->] (0) edge node {$\Open{x}$} (1); \draw[->] (1) edge
				node {$t$} (2); \draw[->] (2) edge node {$h$} (3); \draw[->] (3)
				edge node {$a$} (4); \draw[->] (4) edge node {$t$} (5);
				\draw[->] (5) edge node {$\Close{x}$} (6);
			\end{scope}
		\end{tikzpicture}
	\end{center}
	In this figure, the states are $\{0, \ldots, 5\}$, where $0$ and $5$ are the
	initial and final state, respectively.  The edges between states are
	transitions, where the first and last edges are variable transitions, i.e.,
	they open and close $x$ with the variable markers $\Open{x}$ and
	$\Close{x}$, respectively, and the middle edges are letter transitions.
\end{example}
Example~\ref{example:logicalVA} shows how to compile a REQL query into a logical
VA. Using a construction similar to Thomson's~\citep{HopcroftU79}, we can covert
every REQL query into a logical VA, giving us a logical plan for the query.
\begin{proposition} \label{prop:logicalVA} For every REQL query $\texttt{e}$,
	one can build in linear time a logical VA $\cA$  such that
	$\sem{\texttt{e}}_d = \sem{\cA}_d$ for every document~$d$.
\end{proposition}
\begin{proof}
	Let us first define an automaton that is more closely related to the REQL
	query, this will serve as an itermediary step between the query and the
	logical VA. An $\varepsilon$-logical VA is a logical VA extended with
	$\varepsilon$-transitions. Formally, it is a tuple $\cA = (Q, \delta, q_0,
	q_f)$ defined in the same way as a logical VA, with the difference that the
	transition relation $\delta$ now can additionally store
	$\varepsilon$-transitions of the form $(q, \varepsilon, q')$ with $q, q' \in
	Q$.

	A run $\rho$ of an $\varepsilon$-logical VA $\cA$ over a document $d$ is
	defined in in almost the exact same way as in a logical VA, with the
	consideration that an $\varepsilon$-transition acts like an empty variable
	marker; that is, (1) the automaton doesn't move on the document when an
	$\varepsilon$-transition occurs, and (2) the mapping $\mu^\rho$ is not
	defined in terms of the $\varepsilon$-transitions that occurred on $\rho$.
	The concepts of accepting run and semantics over a document of an
	$\varepsilon$-logical VA are defined in the same way as in a logical VA.

	It not hard to see that an $\varepsilon$-logical VA $\cA$ can be build into
	a logical VA $\cA'$ in linear time such that $\semd{\cA} = \semd{\cA'}$ for
	any document $d$ if $\varepsilon \notin \semd{\cA}$. We start by taking the
	$\varepsilon$-closure for each state $q$ of $\cA$ (i.e., the set of states
	that can be reached from $q$ by only traversing $\varepsilon$-transitions),
	and then for each $(r, o, s) \in \delta$ such that $r$ is in the
	$\varepsilon$-closure of $q$ and $o \neq \varepsilon$, we say that $(q, o,
	s)$ is in the new transition relation $\delta'$. We then take the reverse
	$\varepsilon$-closure of $q_f$ (the set of states that can reach $q_f$ by
	only traversing $\varepsilon$-transitions) and, for each $(s, o, r) \in
	\delta$ such that $r$ is in the reverse $\varepsilon$-closure of $q_f$ and
	$o \neq \varepsilon$, we say that $(s, o, q_f)$ is in the new transition
	relation $\delta'$. Given that $\varepsilon \notin \semd{\cA}$, we can be
	sure that the logical VA $\cA' = (Q, \delta', q_0, q_f)$ satisfies
	$\semd{\cA} = \semd{\cA'}$ for any document $d$, given that $q_f$ not being
	in the $\varepsilon$-closure of $q_0$ doesn't change $\semd{\cA}$.

	Let $\texttt{e}$ be a REQL. Then, $\varepsilon \notin \semd{\texttt{e}}$ for
	any document $d$. We will show that there exists an $\varepsilon$-LogicalVA
	$\cA$ such that $\sem{\texttt{e}}_d = \sem{\cA}_d$ over any document $d$. We
	proceed by induction. For each of the base cases shown in Table
	\ref*{tab-semantics} we can construct a LogicalVA $\cA$ with two states: one
	initial and one accepting, and with a single transition labeled as the char
	class equivalent to $\texttt{e}$ from the initial state to the accepting
	state. Is easy to see that this LogicalVA $\cA$ satisfies
	$\sem{\texttt{e}}_d = \sem{\cA}_d$ over any document $d$.

	The inductive case goes as follows: suppose that $\texttt{e}_1$ and
	$\texttt{e}_2$ are two REQLs such that there exist $\cA_1 = (Q_1, \delta_1,
	q_0^1, q_f^1)$ and $\cA_2 = (Q_2, \delta_2, q_0^2, q_f^2)$ both LogicalVAs
	that satisfy $\sem{\texttt{e}_1}_d = \sem{\cA_1}_d$ and
	$\sem{\texttt{e}_2}_d = \sem{\cA_2}_d$ over any document $d$. Then,

	\begin{itemize}
		\item For $\texttt{e} = \texttt{e}_1\texttt{e}_2$ the LogicalVA $\cA =
			      (Q_1 \cup Q_2, \delta_1 \cup \delta_2 \cup \{(q_f^1,
			      \varepsilon, q_0^2)\}, q_0^1, q_f^2)$ is such that over any
			      document $d$, $\sem{\texttt{e}}_d = \sem{\cA}_d$.
		\item For $\texttt{e} = \texttt{e}_1\texttt{|}\texttt{e}_2$ the
		      LogicalVA $\cA = (Q', \delta', q_0', q_f')$ with
		      \begin{align*}
			      Q'      & = Q_1 \cup Q_2 \cup \{q_0, q_f\}                                                                                                              \\
			      \delta' & = \delta_1 \cup \delta_2 \cup \{(q_0, \varepsilon, q_0^1), (q_0, \varepsilon, q_0^2), (q_f^1, \varepsilon, q_f), (q_f^2, \varepsilon, q_f) \} \\
			      q_0'    & = q_0                                                                                                                                         \\
			      q_f'    & = q_f
		      \end{align*}
		      is such that over any document $d$, $\sem{\texttt{e}}_d =
			      \sem{\cA}_d$.
		\item For $\texttt{e} = \texttt{!x\{}\texttt{e}_1\texttt{\}}$ the
		      LogicalVA $\cA$ is defined as
		      \[
			      \cA = (Q' \cup \{q_0, q_f\},\delta' \cup \{(q_0, [x, q_0'), (q_f', x\rangle, q_f)\}, q_0, q_f)
		      \]
		      such that over any document $d$, $\sem{\texttt{e}}_d =
			      \sem{\cA_\texttt{e}}_d$.
		\item For $\texttt{e} = \texttt{e}_1\texttt{*}$ the LogicalVA $\cA$
		      defined as
		      \[
			      \cA = (Q_1 \cup \{q_0', q_f'\},
			      \delta_1 \cup \{(q_0', \varepsilon, q_0^1), (q_f^1, \varepsilon, q_0^1), (q_f^1, \varepsilon, q_f'), (q_0', \varepsilon, q_f')\},
			      q_0', q_f')
		      \]
		      is such that over any document $d$, $\sem{\texttt{e}}_d =
			      \sem{\cA}_d$.
		\item For $\texttt{e} = \texttt{e}_1\texttt{?}$ the LogicalVA $\cA$
		      defined as
		      \[
			      \cA = (Q_1 \cup \{q_0', q_f'\},
			      \delta_1 \cup \{(q_0', \varepsilon, q_0^1), (q_f^1, \varepsilon, q_f'), (q_0', \varepsilon, q_f')\},
			      q_0', q_f')
		      \]
		      is such that over any document $d$, $\sem{\texttt{e}}_d =
			      \sem{\cA}_d$.
		\item For $\texttt{e} = \texttt{e}_1\texttt{+}$, we have that
		      $\semd{\texttt{e}_1\texttt{+}} =
		      \semd{\texttt{e}_1(\texttt{e}_1\texttt{*})}$ for any document $d$.
		      This case is then inductive as it can be constructed from previous
		      rules.
		\item For $\texttt{e} = \texttt{e}_1\texttt{\{n,m\}}$ we have that
		      $\sem{\texttt{e}}_d =
		      \sem{(\texttt{e}_1)^n(\texttt{e}_1\texttt{?})^{m-n}}_d$ for any
		      document $d$. This case is then inductive as it can be constructed
		      from previous rules.
	\end{itemize}

\end{proof}
Note that logical VA is an extension of \emph{variable-set automata} (VA)
from~\citet{FaginKRV15}. The main difference between the two models is that
logical VA uses char classes in its letter transitions whereas VA uses
individual letters. Moreover, a logical VA can start a run at any position,
whereas VA starts from the beginning of the document. Although both models are
equally expressive, we use logical VAs as logical plans for compiling REQL
formulas in practice.

\section{Offsets (optimization)}

In some cases, opening a variable can be postponed in order not to store the
information about runs that will not result in an output. To illustrate this,
consider again the expression \texttt{e0} and its logical VA of
Example~\ref{example:logicalVA}. Intuitively, our algorithm needs to store the
position information for the opening of a variable $x$ every time a \texttt{t}
would be read. If the document we are reading has the text \texttt{thasty}, this
run would then be extended for two more steps, although it will eventually be
abandoned, and not result in any outputs. In cases such as these, we can
actually postpone (offset) opening of the variable $x$ by transforming the
logical VA as follows: (i) first read the word \texttt{that}; (ii) now open a
variable marker $\Open{x}$, but remember that it was actually opened four
symbols before (i.e., it has an offset 4); (iii) proceed with the current run.
When reconstructing the output, we will start reading four symbols before the
position that is stored for $\Open{x}$. The transformation of the logical VA
from Example~\ref{example:logicalVA} would look as follows:
\begin{center}
	\begin{tikzpicture}[->,>=stealth, semithick, auto, initial text= {}, initial distance= {3mm}, accepting distance= {4mm}, node distance=1.7cm, semithick]
		\tikzstyle{every state}=[draw=black,text=black,inner sep=0pt, minimum
		size=7mm]

		\begin{scope}
			\node[initial,state] (0) {$0$}; \node[state] (1) [right of=0] {$1$};
			\node[state] (2) [right of=1] {$2$}; \node[state] (3) [right of=2]
			{$3$}; \node[state] (4) [right of=3] {$4$}; \node[state] (5) [right
			of=4] {$5$}; \node[state,accepting] (6) [right of=5] {$6$};

			\draw[->] (0) edge node {$t$} (1); \draw[->] (1) edge node {$h$}
			(2); \draw[->] (2) edge node {$a$} (3); \draw[->] (3) edge node
			{$t$} (4); \draw[->] (4) edge node {$\Open{x}^{-4}$} (5); \draw[->]
			(5) edge node {$\Close{x}$} (6);
		\end{scope}
	\end{tikzpicture}
\end{center}
The notation $\Open{x}^{-4}$ is used in order to signal that the variable $x$
was actually opened four positions before it was recorded in the run. %when
opened, the variable \texttt{x} actually stores characters starting four
positions before the guarded position.

% Offsets are implemented in the rewriting module of \rematch\ after
% constructing the first logical VA from a REQL query. When the query contains
% quantifiers (\texttt{*}, \texttt{+}, \texttt{?} and \texttt{\{$n$,$m$\}}
% operators) or alternations (\texttt{|} operator) special care must be taken
% for offsetting the variables. More details are provided in Appendix
% \ref{app:offset}.

\subsection{Offset logical VA} 
To formalize this concept we extend the definition of a logical VA as follows: a
\emph{logical variable-set automata} (offset logical VA) $\cAoff$ is a tuple
$(Q, \delta, \tau, q_0, q_f)$ that has the same structure as a logical VA, with
the addition of an \emph{offset marker} function $\tau$ that takes a variable
marker as input and returns an integer.

A configuration, a run and an accepting run of $\cAoff$ are defined exactly as
if $\cAoff$ were a logical VA. The only change is in the definition of its
semantics. Suppose that 
$$
	\rho \ = \ (q_0, i_0) \ \trans{o_1} \ (q_1, i_1) \ \trans{o_2} \ \cdots \ \trans{o_m} \ (q_m, i_m)
$$
is an accepting run of $\cAoff$. The mapping $\mu^\rho$ then is such that it
maps $x$ to $[i_j, i_k\rangle$ if, and only if, $o_{i_j + \tau([x)} = [x$ and
$o_{i_k + \tau(x\rangle)} = x\rangle$. In other words, the mapping $\mu^\rho$
acts in the same way as if $\cAoff$ were a logical VA, but taking into account
the offsets given by $\tau$ for each of the variable markers present in the
transitions of $\cAoff$. Therefore, the semantics of $\sem{\cAoff}_d$ of
$\cAoff$ over a document $d$ is defined as the set of all $\mu^\rho$ where
$\rho$ is an accepting run of $\cAoff$ over $d$.

Given the way Algorithm \ref*{alg:evaluation} works, it is desirable to be able
to construct an offset logical VA $\cAoff$ from an initial logical VA $\cA$ such
that: (i) the semantics are preserved, i.e., $\sem{\cAoff}_d = \sem{\cA}_d$, and
(ii) the offset function $\tau$ of $\cAoff$ is optimal in the sense of
maximizing the absolute values of the offsets for the variable markers, so that
the decision of changing the state of the ECS is postponed as much as possible
while reading the document. In REmatch, this is done by the following algorithm,
which we call \texttt{offset}.

\subsection{Offset algorithm}
In the implementation of REmatch, the \texttt{offset} optimization is done in
the early stages of preprocessing the query.

First, for each variable opening we build a list of all the transitions that
open this variable, and we do the same for each variable closing. The variable
opening/closing is then offset in bulk, namely, either all of the transitions of
the form $\Open{x}$ (or $\Close{x}$) are moved at once (in order to preserve
consistency), or none is. Let \texttt{captureList} be a list of all capture
transitions opening or closing some variable (for instance, they are all of the
form $\Open{x}$). For a capture transition $p\ \longtrans{\Open{x}}\ q$ we wish
to see if there is also a transition $q\ \longtrans{a}\ r$, in order to
interchange the transition reading the letter $a$, and the one opening the
variable $x$. That is, we wish to achieve the following transformation in the
states of our eVA:
$$p\ \longtrans{\Open{x}}\ q\ \longtrans{a}\ r \ \ \ \ \ \Rightarrow \ \ \ \ \ \
\ p\ \longtrans{a}\ q\ \longtrans{\Open{x}}\ r$$


We will say that we can offset $\Open{x}$ if the following conditions hold for
every transition $p\ \longtrans{\Open{x}}\ q$ appearing in \texttt{captureList}
associated with $\Open{x}$:
\begin{enumerate}
	\item $q$ is not a final state;
	\item There are no transitions of the form $q\ \longtrans{\Open{y}}\ r$, or
	of the form $p'\ \longtrans{v}\ q$, with $v$ a variable marker, in the
	automaton; \label{item:offset-cond-2}
	\item There is at least one transition of the form $q\ \longtrans{a}\ r$;
	\item For all transitions $q\ \longtrans{a}\ r$, $q$ can not be reached from
	$r$; and
	\item For any other transition $p'\ \longtrans{\Open{x}}\ q'$, if we take
	any $q\ \longtrans{a}\ r$, then $q'$ is not reachable from $r$.
\end{enumerate}

The first condition prevents offsetting a variable that leads to an accepting
run. The second condition prevents manipulating a state which has multiple
capture transitions associated with it. The third condition assures we actually
have a transition to offset. Fourth transition makes sure that in case of moving
the $\Open{x}$ variable marker forward, we will not create any loops involving
this variable marker. Finally, the last condition ensures that moving one
$\Open{x}$ variable marker will not result in an inconsistent run involving
another such transition. This could, for instance, happen if we had both $p\
\longtrans{\Open{x}}\ q\ \longtrans{a}\ r$ and $p'\ \longtrans{\Open{x}}\ q'\
\longtrans{a}\ q$ in our automaton, since offsetting the first $\Open{x}$
transition would result in an automaton that has a run opening $x$ twice.

Given each such list \texttt{captureList} (for example for $\Open{x}$), we
manipulate each set of transitions $p\ \longtrans{\Open{x}}\ q\ \longtrans{a}\
r$, by switching the $\Open{x}$ and $a$ symbols (in reality, an auxiliary state
is created for this). The new \texttt{captureList} is then populated by adding
the transitions $q\ \longtrans{\Open{x}}\ r$ to it, and the process is repeated
as long as the newly created list satisfies the five conditions specified above.

Finally, we process the variable offsets by finding the reverse topological
order of all the capture transitions. Then we offset the variables in this order
so that we can move them as far forward as possible. To illustrate why this is
important, consider the automaton consisting of a single run as follows:
$$q_0\ \longtrans{\Open{x}}\ q_1\ \longtrans{a}\ q_2\ \longtrans{\Close{x}}\
q_3\ \longtrans{b}\ q4$$ If we were to first offset the $\Open{x}$ variable
marker, this would result in:
$$q_0\ \longtrans{a}\ q_1\ \longtrans{\Open{x}^{-1}}\ q_2\
\longtrans{\Close{x}}\ q_3\ \longtrans{b}\ q4.$$ Notice that $\Open{x}$ cannot
be not be further offset due to condition (ii) above. Then, if we offset
$\Close{x}$ we get: $$q_0\ \longtrans{a}\ q_1\ \longtrans{\Open{x}^{-1}}\ q_2\
\longtrans{b}\ q_3\ \longtrans{\Close{x}^{-1}}\ q4$$ Which is suboptimal as
$\Open{x}^{-1}$ can still be offset from its position in this new automaton. If
instead the variables are processed in reverse topological order, this results
in:
$$q_0\ \longtrans{a}\ q_1\ \longtrans{b}\ q_2\ \longtrans{\Open{x}^{-2}}\ q_3\
\longtrans{\Close{x}^{-1}}\ q4$$ which is better in terms of the maximization of
the absolute values of the offsets.
