% % !TeX root = ../main.tex

Regular expressions, or RegEx, are one of the most used technologies for
managing text data. The development of RegEx engines started in the early
70s~\cite{thompson1968programming,earlyNFA}, and they are now a common part of
many complex information systems such as compilers, databases, or search
engines. Moreover, modern RegEx engines are highly-optimized systems that are
crucial for finding patterns in diverse areas like biology~\cite{Gonzalo},
literature~\cite{lit1}, or medicine~\cite{FloresFP21}. 


Given a regular expression and a document, the task of a RegEx engine is to find
all occurrences, or \emph{matches}, of the pattern in the document. For this,
RegEx engines deploy the so-called \emph{leftmost-longest}
paradigm~\cite{posix}, meaning that they find the match which is the leftmost
one, and from there they find the longest possible match. The process is then
repeated  starting from the rightmost position of the previous
match\footnote{Although RegEx engines follow different matching rules, the
leftmost-longest rule is at the core of most modern engines. For a detailed
discussion see \cite{friedl2006mastering}.}. For example, if we want to evaluate
the RegEx $\texttt{aa}$ over the document $a_0 a_1 a_2 a_3$ (here the subindices
are for referencing positions; the document consists of the letter $a$ repeated
four times), a typical RegEx engine will output the matches $a_0a_1$ and
$a_2a_3$. In particular, RegEx engines will not output $a_1a_2$ since the first
leftmost-longest match ends with $a_1$. 

The leftmost-longest semantics is standard for  RegEx engines, as it captures
the majority of meaningful matches, although not all of them. However, in some
scenarios adopting an ``all-match semantics'' is a valuable and desirable
feature for the users. For instance, in DNA analysis we will often need to match
patterns (called motifs) onto a DNA sequence, and these can overlap. The
question of finding overlapping matches with RegEx is also recurrent in user
discussions~\cite{overlap1,overlap2,overlap3}. For information extraction, the
all-match semantics leaves freedom to the user to extract all positions, called
spans, where there is relevant information in a document. Therefore the
all-match semantics is a desirable feature for RegEx engines that, to the best
of our knowledge, no engine supports natively.

To overcome the problem of finding all-matches, RegEx engines offer look-around
operators, namely, operators that allows checking if a subexpression can be
matched forward or backward from the current position, without advancing from
the current position. For instance, by using look-around, we can modify the
expression \texttt{aa} to \texttt{(?=(aa))} and find the missing match $a_1a_2$
over the above document. Despite this example, look-around operators cannot
discover all matches for every RegEx expression. For instance, given the
look-around definition, one cannot extract two matches that start at the same
position (for concrete examples see Section~\ref{chpt:regex} and
Section~\ref{chpt:experiments}).

In terms of implementation, RegEx engines are usually divided into three
categories: DFA-based, NFA-based, and recursive NFA-based~\cite{cox2007regular}.
DFA is generally the fastest evaluation strategy, followed by (plain) NFA. In
contrast, recursive NFA-based engines use backtracking, which is susceptible to
well-documented performance issues, like regular expression denial of service
attacks (ReDos)~\cite{friedl2006mastering}, where the engine can exhibit
exponential time performance~\cite{cox2007regular}. From the positive side,
recursive NFA-based engines have the advantage of keeping track of the
evaluation, which allows implementing operators like look-around and
back-references. In summary, until now, the only way of finding all matches (in
some cases) is by using look-around operators implemented by recursive NFA-based
engines, which suffer from unfortunate performance issues. 

To overcome these issues, this thesis presents \rematch, a RegEx engine
supporting the all-match semantics, and its accompanying regular expression
language REQL. Contrary to the status quo of RegEx evaluation, \rematch\ is
based on a new evaluation strategy, inspired by the theory of enumeration
algorithms~\cite{Segoufin13}, that allows finding all the matches, and avoids
the exponential behavior of recursive NFA evaluation. Moreover, \rematch\
performance is comparable to popular RegEx engines, while at the same time
finding all the matches, thus obtaining the best of both worlds. Specific
contributions of the thesis are as follows:

\begin{enumerate}

\item  We introduce the REQL query language, which extends classical RegEx with
variables and the all-match semantics.	

\item We present \rematch, a RegEx system whose architecture allows evaluating
REQL using output-linear delay. For this, we develop a new evaluation method
which extends the theoretical algorithm of~\cite{FlorenzanoRUVV20} and
incorporates new optimization techniques, allowing \rematch\ to compete with
modern RegEx engines. 

\item We develop a set of experiments to evaluate the effect of different
optimizations on \rematch\ performance, and compare it to existing RegEx
engines. Although \rematch\ uses a more general semantics, we show that its
performance stacks well compared to other engines.
	
\end{enumerate}

In Chapter~\ref{chpt:regex} we introduce REQL. We then explain each module of
the \rematch\ architecture (see~Figure~\ref{fig:architecture}).
Chapter~\ref{chpt:rewriting} presents the rewriting module,
Chapter~\ref{chpt:filtering} the filtering module, and Chapter~\ref{chpt:output}
the output module. Chapter~\ref{chpt:evaluation} explains the evaluation
algorithm of \rematch. Chapter~\ref{chpt:experiments} puts all components
together and displays the experimental comparison with other engines. We
conclude in Chapter~\ref{chpt:conclusions} by discussing possible future work. 
