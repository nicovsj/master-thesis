% % !TeX root = ../main.tex

Regular expressions, or RegEx, are one of the most used technologies for
managing text data. RegEx engines, integral to many complex information systems
like compilers, databases, and search engines, have evolved significantly since
their inception in the late 1960s~\citep{thompson1968programming,earlyNFA}. The
development of these engines paved the way for their widespread use. Classic
Unix programs such as grep, find, sed, and AWK exemplify this by utilizing RegEx
as a core feature. Furthermore, modern RegEx engines have become highly
optimized, playing a crucial role in pattern recognition across various
domains.~\citep{Gonzalo}, literature~\citep{lit1}, or medicine~\citep{FloresFP21}. 


Given a regular expression and a document, the task of a RegEx engine is to find
all occurrences, or \emph{matches}, of any string in the language of the regular
expression that is also present in the document. For this, RegEx engines deploy
the so-called \emph{leftmost-longest} paradigm~\citep{posix}, meaning that they
find the match whose starting position is the leftmost one, and from there they
find the longest possible match. The process is then repeated  starting from the
rightmost position of the previous match\footnote{Although RegEx engines follow
different matching rules, the leftmost-longest rule is at the core of most
modern engines. For a detailed discussion see \cite{friedl2006mastering}.}. For
example, if we want to evaluate the RegEx $\texttt{aa}$ over the document $a_0
a_1 a_2 a_3$ (here the subindices are for referencing positions; the document
consists of the letter $a$ repeated four times), a typical RegEx engine will
output the matches $a_0a_1$ and $a_2a_3$. In particular, RegEx engines will not
output $a_1a_2$ since the first leftmost-longest match ends with $a_1$. 

The leftmost-longest semantics is standard for RegEx engines, as it captures the
majority of meaningful matches, although not all of them. However, in some
scenarios adopting an ``all-match semantics'' is a valuable and desirable
feature for the users. For instance, in DNA analysis we will often need to match
patterns (called motifs) onto a DNA sequence, and these can overlap. The
question of finding overlapping matches with RegEx is also recurrent in user
discussions~\citep{overlap1,overlap2,overlap3}. For information extraction, the
all-match semantics leaves freedom to the user to extract all positions, called
spans, where there is relevant information in a document. Therefore the
all-match semantics can be a desirable feature for RegEx engines and, to the
best of our knowledge, no engine supports natively.


RegEx engines use special operators called ``look-around" and  that can help to
find all possible matches in a text. These operators check for a match either
ahead or behind the current position in the text, without moving (i.e. consuming
the text) from that spot. For example, if we change the RegEx \texttt{aa} to a
look-ahead (a type of look-around) with \texttt{(?=aa)}, we can find the missing
match $a_1a_2$ from the example above. This is because the look-ahead acts an
assertion in the RegEx that can capture the string $aa$ without consuming it in
the text, and thus, opening the possibility to capture overlapping matches.
However, even with look-around, it's not always possible to find every match.
Specifically, look-around can't find multiple matches that begin at the same
point in the text.  (for concrete examples see Sections~\ref{chpt:regex} and
\ref{chpt:experiments}).

In terms of implementation, RegEx engines are usually divided into three
categories: DFA-based, NFA-based, and recursive NFA-based~\citep{cox2007regular}.
DFA is generally the fastest evaluation strategy (if the RegEx is small enough),
followed by (plain) NFA. In contrast, recursive NFA-based engines use
backtracking, which is susceptible to well-documented performance issues, like
regular expression denial of service attacks (ReDos)~\citep{friedl2006mastering},
where the engine can exhibit exponential time performance~\citep{cox2007regular}.
From the positive side, recursive NFA-based engines have the advantage of
keeping track of the evaluation, which allows implementing operators like
look-around and back-reference (an operator that allows matching the same text
as previously matched by a capturing group), and also can avoid worst-cases
where the DFA is exponential in the size of the RegEx. In summary, until now,
the only way of finding all matches (in some cases) is by using look-around
operators implemented by recursive NFA-based engines, which suffer from
unfortunate performance issues. 

To overcome these issues, this thesis presents \rematch, a RegEx engine
supporting the all-match semantics, and its accompanying regular expression
language REQL. Contrary to the status quo of RegEx evaluation, \rematch\ is
based on a new evaluation strategy, inspired by the theory of enumeration
algorithms~\citep{Segoufin13}, that allows finding all the matches, and avoids
the exponential behavior of recursive NFA evaluation. Moreover, \rematch\
performance is comparable to popular RegEx engines, while at the same time
finding all the matches, thus obtaining the best of both worlds. Specific
contributions of the thesis are as follows:

\begin{enumerate}

\item  We introduce the REQL query language, which extends classical RegEx with
variables and the all-match semantics.	

\item We present \rematch, a RegEx system whose architecture allows evaluating
REQL using output-linear delay. For this, we develop a new evaluation method
that extends the theoretical algorithm of~\citep{FlorenzanoRUVV20} and
incorporates new optimization techniques, allowing \rematch\ to compete with
modern RegEx engines. 

\item We develop a set of experiments to evaluate the effect of different
optimizations on \rematch\ performance, and compare it to existing RegEx
engines. Although \rematch\ uses a more general semantics, we show that its
performance stacks well compared to other engines.
	
\end{enumerate}

In Chapter~\ref{chpt:regex} we introduce REQL. We then explain each module of
the \rematch\ architecture (see~Figure~\ref{fig:architecture}).
Chapter~\ref{chpt:rewriting} presents the rewriting module,
Chapter~\ref{chpt:filtering} the filtering module, and Chapter~\ref{chpt:output}
the output module. Chapter~\ref{chpt:evaluation} explains the evaluation
algorithm of \rematch. Chapter~\ref{chpt:experiments} puts all components
together and displays the experimental comparison with other engines. We
conclude in Chapter~\ref{chpt:conclusions} by discussing possible future work. 
