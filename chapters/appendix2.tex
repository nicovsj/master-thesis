% !TeX root = ../Thesis.tex

In some cases, opening a variable can be postponed in order to avoid the storage
of the information about runs that will not result in an output. To illustrate
this, consider the expression
\begin{center}
	\texttt{!x\{sparql[\textasciicircum \textbackslash n]*\}\textbackslash n}
\end{center}
which looks for the keyword \texttt{sparql}, and then proceeds to capture the
text until the first end of line symbol is reached. In a log file, such as the
ones studied in~\citet{AMRV16}, this roughly corresponds to capturing a SPARQL
query. Notice that here Algorithm \ref{alg:evaluation} would store the position
information for the opening of a variable \texttt{x} every time an \texttt{s}
would be read. If the document we are reading has the text \texttt{sparx}, this
run would then be extended for three more steps, although it will eventually be
abandoned, and not result in any outputs. Given the node management required
when changing the state of the ECS structure, this can bring a significant
overhead to evaluation. In cases such as these, we can actually postpone (i.e.,
offset) the opening of the variable \texttt{x} by proceeding as follows: (i)
first read the word \texttt{sparql}; (ii) now open a variable \texttt{x}, but
remember that it was actually opened six symbols before (i.e. it has an offset
6); (iii) proceed until the end of the expression. Then, when reconstructing the
output in case of a successful run, we will simply start reading the output six
symbols before the position that is actually stored in \texttt{ds} from
Algorithm \ref{alg:evaluation}. Intuitively, one could write the expression
above as:
\begin{center}
	\texttt{sparql!x\{$^{-6}$[\textasciicircum \textbackslash n]*\}
	\textbackslash n} 
\end{center}
in order to signal that when opened, the variable \texttt{x} actually stores
characters starting six positions before the guarded position.

\subsection{Offset logical VA} 
To formalize this concept we extend the definition of a logical VA as follows: a
\emph{logical variable-set automata} (offset logical VA) $\cAoff$ is a tuple
$(Q, \delta, \tau, q_0, q_f)$ that has the same structure as a logical VA, with
the addition of an \emph{offset marker} function $\tau$ that takes a variable
marker as input and returns an integer.

A configuration, a run and an accepting run of $\cAoff$ are defined exactly as
if $\cAoff$ were a logical VA. The only change is in the definition of its
semantics. Suppose that 
$$
	\rho \ = \ (q_0, i_0) \ \trans{o_1} \ (q_1, i_1) \ \trans{o_2} \ \cdots \ \trans{o_m} \ (q_m, i_m)
$$
is an accepting run of $\cAoff$. The mapping $\mu^\rho$ then is such that it
maps $x$ to $[i_j, i_k\rangle$ if, and only if, $o_{i_j + \tau([x)} = [x$ and
$o_{i_k + \tau(x\rangle)} = x\rangle$. In other words, the mapping $\mu^\rho$
acts in the same way as if $\cAoff$ were a logical VA, but taking into account
the offsets given by $\tau$ for each of the variable markers present in the
transitions of $\cAoff$. Therefore, the semantics of $\sem{\cAoff}_d$ of
$\cAoff$ over a document $d$ is defined as the set of all $\mu^\rho$ where
$\rho$ is an accepting run of $\cAoff$ over $d$.

Given the way Algorithm \ref*{alg:evaluation} works, it is desirable to be able
to construct an offset logical VA $\cAoff$ from an initial logical VA $\cA$ such
that: (i) the semantics are preserved, i.e., $\sem{\cAoff}_d = \sem{\cA}_d$, and
(ii) the offset function $\tau$ of $\cAoff$ is optimal in the sense of
maximizing the absolute values of the offsets for the variable markers, so that
the decision of changing the state of the ECS is postponed as much as possible
while reading the document. In REmatch, this is done by the following algorithm,
which we call \texttt{offset}.

\subsection{Offset algorithm}
In the implementation of REmatch, the \texttt{offset} optimization is done in
the early stages of preprocessing the query. After the first parse of the
expression, REmatch will build an extended variable automaton (eVA) as described
in Section \ref{sec:extendedVA}.

First, for each variable opening we build a list of all the transitions that
open this variable, and we do the same for each variable closing. The variable
opening/closing is then offset in bulk, namely, either all of the transitions of
the form $\Open{x}$ (or $\Close{x}$) are moved at once (in order to preserve
consistency), or none is. Let \texttt{captureList} be a list of all capture
transitions opening or closing some variable (for instance, they are all of the
form $\Open{x}$). For a capture transition $p\ \longtrans{\Open{x}}\ q$ we wish
to see if there is also a transition $q\ \longtrans{a}\ r$, in order to
interchange the transition reading the letter $a$, and the one opening the
variable $x$. That is, we wish to achieve the following transformation in the
states of our eVA:
$$p\ \longtrans{\Open{x}}\ q\ \longtrans{a}\ r \ \ \ \ \ \Rightarrow \ \ \ \ \ \
\ p\ \longtrans{a}\ q\ \longtrans{\Open{x}}\ r$$


We will say that we can offset $\Open{x}$ if the following conditions hold for
every transition $p\ \longtrans{\Open{x}}\ q$ appearing in \texttt{captureList}
associated with $\Open{x}$:
\begin{enumerate}
	\item $q$ is not a final state;
	\item There are no transitions of the form $q\ \longtrans{\Open{y}}\ r$, or
	of the form $p'\ \longtrans{v}\ q$, with $v$ a variable marker, in the
	automaton; \label{item:offset-cond-2}
	\item There is at least one transition of the form $q\ \longtrans{a}\ r$;
	\item For all transitions $q\ \longtrans{a}\ r$, $q$ can not be reached from
	$r$; and
	\item For any other transition $p'\ \longtrans{\Open{x}}\ q'$, if we take
	any $q\ \longtrans{a}\ r$, then $q'$ is not reachable from $r$.
\end{enumerate}

The first condition prevents offsetting a variable that leads to an accepting
run. The second condition prevents manipulating a state which has multiple
capture transitions associated with it. The third condition assures we actually
have a transition to offset. Fourth transition makes sure that in case of moving
the $\Open{x}$ variable marker forward, we will not create any loops involving
this variable marker. Finally, the last condition ensures that moving one
$\Open{x}$ variable marker will not result in an inconsistent run involving
another such transition. This could, for instance, happen if we had both $p\
\longtrans{\Open{x}}\ q\ \longtrans{a}\ r$ and $p'\ \longtrans{\Open{x}}\ q'\
\longtrans{a}\ q$ in our automaton, since offsetting the first $\Open{x}$
transition would result in an automaton that has a run opening $x$ twice.

Given each such list \texttt{captureList} (for example for $\Open{x}$), we
manipulate each set of transitions $p\ \longtrans{\Open{x}}\ q\ \longtrans{a}\
r$, by switching the $\Open{x}$ and $a$ symbols (in reality, an auxiliary state
is created for this). The new \texttt{captureList} is then populated by adding
the transitions $q\ \longtrans{\Open{x}}\ r$ to it, and the process is repeated
as long as the newly created list satisfies the five conditions specified above.

Finally, we process the variable offsets by finding the reverse topological
order of all the capture transitions. Then we offset the variables in this order
so that we can move them as far forward as possible. To illustrate why this is
important, consider the automaton consisting of a single run as follows:
$$q_0\ \longtrans{\Open{x}}\ q_1\ \longtrans{a}\ q_2\ \longtrans{\Close{x}}\
q_3\ \longtrans{b}\ q4$$ If we were to first offset the $\Open{x}$ variable
marker, this would result in:
$$q_0\ \longtrans{a}\ q_1\ \longtrans{\Open{x}^{-1}}\ q_2\
\longtrans{\Close{x}}\ q_3\ \longtrans{b}\ q4.$$ Notice that $\Open{x}$ cannot
be not be further offset due to condition (ii) above. Then, if we offset
$\Close{x}$ we get: $$q_0\ \longtrans{a}\ q_1\ \longtrans{\Open{x}^{-1}}\ q_2\
\longtrans{b}\ q_3\ \longtrans{\Close{x}^{-1}}\ q4$$ Which is suboptimal as
$\Open{x}^{-1}$ can still be offset from its position. If instead the variables
are processed in reverse topological order, this results in:
$$q_0\ \longtrans{a}\ q_1\ \longtrans{b}\ q_2\ \longtrans{\Open{x}^{-2}}\ q_3\
\longtrans{\Close{x}^{-1}}\ q4$$ which is better in terms of the maximization of
the absolute values of the offsets.