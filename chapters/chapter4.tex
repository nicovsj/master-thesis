% !TeX spellcheck = en_US
% !TeX root = ../Thesis.tex

The output of a REQL query $\texttt{e}$ over a document $d$ can be prohibitively large, namely, of size $\mathcal{O}(|d|^{2|e|})$, since for the all-match semantics we could even ask for all substrings being matched to all the variables. Since such queries are, at least in principle, expressible in REQL, we need to be able to handle them in \rematch. For this, we deploy the notion of enumeration algorithms and output delay, which measure the efficiency of an algorithm  %to process the input and deliver the output.
with respect to both its input and its output.

%As it is standard in the literature~\cite{bagan2006mso,courcelle2009linear,Segoufin13,segoufin2014glimpse}, we will use the RAM model of computation where the space used to store integers, variable markers, and time for memory lookups and update time are all $\mathcal{O}(1)$. 
%To gauge the efficiency of our approach, 
More formally, we use the framework of \emph{enumeration algorithms}, which received a lot of attention in the database community~\cite{AmarilliBMN-tods21,FlorenzanoRUVV20,LosemannM-lics14,SchweikardtSV-jacm22,BerkholzGS-siglog20,Segoufin13,IdrisUVVL-vldbj20,IdrisUV-sigmod17,TziavelisAGRY-pvldb20}. In enumeration algorithms, we are required to produce the (finite) output set $O = \{o_1,\ldots ,o_k\}$, in any order, and without repetitions. Such algorithms operate in two phases:
\begin{enumerate}
	\item The pre-processing phase builds a data structure which allows enumerating the results;
	\item The enumeration phase retrieves the outputs from the data structure.% without repetitions.
\end{enumerate}
In the case of REQL queries, the desired result is the set of all the output mappings. We will say that an enumeration algorithm works with \emph{output-linear delay}, if the time to print out the $i$-th output $o_i$, measured as the time needed from printing the $(i-1)$-st output $o_{i-1}$ to finishing the output $o_i$, is proportional to the length of~$o_i$, and independent of the size of the document, the query, or the size of the output set $O$. The algorithm is also required to terminate immediately after outputting the final output. If these conditions are met, the time needed to enumerate $O$ is $\mathcal{O}(|O|)$, hence output-linear. 

%Then the purpose of this section is to develop the module in charge of managing the outputs to guarantee output-linear delay and manage the system's memory efficiently.

Next we describe the \rematch\ module for managing the system memory and the data structure supporting output-linear delay.

\section{Enumerable compact set} 
In general, the pre-processing phase builds a  data structure encoding all the mappings that are to be enumerated. We next explain which operations this structure needs to support in order to encode outputs of a REQL query. For this, consider again the document $d_1$ from Figure~\ref{fig-doc} and the REQL query:
$$
\texttt{e4} := \texttt{!x\{th\}.*!y\{hat\}}
$$
%represented by the logical VA $\cA_4$:
%\vspace{-2mm}
%\begin{center}
%	\begin{tikzpicture}[->,>=stealth, semithick, auto, initial text= {}, initial distance= {3mm}, accepting distance= {4mm}, node distance=0.82cm, semithick]
%		\tikzstyle{every state}=[draw=black,text=black,inner sep=0pt, minimum size=4.3mm]
%			\node[initial,state] (0) {$0$};
%			\node[state] (1) [right of=0] {$1$};
%			\node[state] (2) [right of=1] {$2$};
%			\node[state] (3) [right of=2] {$3$};
%			\node[state] (4) [right of=3] {$4$};
%			\node[state] (5) [right of=4] {$5$};
%			\node[state] (6) [right of=5] {$6$};
%			\node[state] (7) [right of=6] {$7$};
%			\node[state] (8) [right of=7] {$8$};
%			\node[state,accepting] (9) [right of=8] {$9$};
%			
%			\draw[->] (0) edge node {$\Open{x}$} (1);
%			\draw[->] (1) edge node {$t$} (2);
%			\draw[->] (2) edge node {$h$} (3);
%			\draw[->] (3) edge node {$\Close{x}$} (4);
%			\draw[->] (4) edge[loop above] node {$.$} (4);
%			\draw[->] (4) edge node {$\Open{y}$} (5);
%			\draw[->] (5) edge node {$h$} (6);
%			\draw[->] (6) edge node {$a$} (7);
%			\draw[->] (7) edge node {$t$} (8);
%			\draw[->] (8) edge node {$\Close{y}$} (9);	
%	\end{tikzpicture}
%\end{center}
that extracts the substring $\texttt{th}$ in the variable $x$ followed by the substring $\texttt{hat}$ in the variable $y$.
One output here is $\mu_1$, with $\mu_1(x) = [0,2\rangle$, and $\mu_1(y) = [4,7\rangle$. Notice that for each output mapping, we need to define when a variable is opened, and when it is closed, in order to define the span it captures. In \rematch\ we will represent a mapping as an \emph{output sequence} of pairs $(S,i)$, where $S$ is a set of variable markers (e.g., $S=\{\Close{x}, \Open{y}\}$), denoting when a span associated with the variable $x$ starts, or finishes, respectively. Therefore, the mapping $\mu_1$ is represented by the output sequence\footnote{Strictly speaking, we should write $(\{\Open{x}\},0),(\{\Close{x}\},2),\ldots$. For the sake of simplification, we omit the set brackets whenever it is possible.}:
$$
(\Open{x},0),(\Close{x},2),(\Open{y},4),(\Close{y},7).
$$
In essence, \rematch\ will create a data structure encoding this information for each mapping. Since many mappings will share information (for instance, $\mu_2$, with $\mu_2(x)=[0,2\rangle$, and $\mu_2(y)=[7,9\rangle$ has the $x$-part identical to $\mu_1$), we can exploit this fact to create a succinct representation of all the outputs.

The data structure we will use to represent the set of all output sequences in $\sem{\texttt{e}}_d$, for a REQL query $\texttt{e}$ and a document $d$ is called \emph{enumerable compact set}, or ECS for short, and was first introduced in~\cite{AmarilliJMR22,MunozR22}. The ECS data structure can be thought as a directed acyclic graph (DAG) with three types of nodes $\coid$:
\begin{itemize}
	\item[(i)] a (unique) terminal node, denoted \cinit \ (or $\bot$ for short), which has no children and tells us that we reached the end of an output;
	\item[(ii)] content nodes, which store a pair $(S,i)$ from an output sequence and have a single child $\coid'$; and
	\item[(iii)] union nodes, which have two children $\coid_1$ and $\coid_2$.
\end{itemize}
More importantly, every node $\coid$ defines a set of mappings $\sem{\coid}$, represented as a set of output sequences. Specifically, the \cinit \ represents the set $\sem{\cinit} = \{\epsilon\}$, the output node $\coid$ with the child $\coid'$ represents the set $\sem{\coid'} \cdot \{(S,i)\}$, and the union node with children $\coid_1$ and $\coid_2$ the set $\sem{\coid_1} \cup \sem{\coid_2}$.
%Certain nodes are then designated as \emph{output nodes}. Output nodes have no incoming edges, and each path starting in an output node and leading to \cinit\ corresponds to an output. Enumeration in ECS is achieved by enumerating all such paths.

%\domagoj{Should we say that the output can be partial? I.e., not everything open was closed? We should then have a theorem saying that the algorithm closes all the open variables?}

\begin{example}
Consider again the query $\texttt{e4}$, and document $d_1$ from Figure~\ref{fig-doc}. The list of all mappings in $\sem{\texttt{e4}}_d$, as represented by their output sequences is:
\begin{align*}
	\mu_1 & \text{: } \ \ (\Open{x},0),(\Close{x},2),(\Open{y},4),(\Close{y},7)  \\% \mu_1(x)=[0,2\rangle, \mu_1(y) = [4,7\rangle\\
	\mu_2 & \text{: } \ \ (\Open{x},0),(\Close{x},2),(\Open{y},7),(\Close{y},10) \\ %\mu_2(x)=[0,2\rangle, \mu_2(y) = [7,10\rangle \\
	\mu_3 & \text{: } \ \ (\Open{x},3),(\Close{x},5),(\Open{y},7),(\Close{y},10)
\end{align*}
The ECS for this set of output sequences is given in Figure~\ref{fig-ecs}. Here the rightmost union node represents the output sequences of $\mu_1$, $\mu_2$, and $\mu_3$. As we can see, following paths from this node to the $\bot$ node represent the three output sequences. The shared output for the variable $y$ in $\mu_2$ and $\mu_3$ is represented only once.
\end{example}

%\paragraph*{Node manager} The key step when evaluating a REQL query over a document is building the ECS which encodes all the outputs. Naively building the ECS would require allocating the memory for each node being added, which is highly inefficient. Therefore, \rematch\ includes a \emph{node manager} module, denoted $\ds$, which allocates memory in bulk, and acts as a garbage collector for the ECS. The $\ds$ module essentially creates a memory pool for storing the nodes in the ECS. Each time $\ds$ fills the pool, it allocates the memory for another pool double the original size. This strategy allows memory allocation to occur infrequently and in big chunks, thus preventing fragmentation and multiple pointer dereferencing. In addition, the $\ds$ module acts as a lazy garbage collector by providing a method $\ds.\texttt{garbage}(\coid)$. This method marks the node $\coid$ from the memory pool that will not be used in the future, preventing the allocation of an additional memory pool if possible. This marking can occur when a node is not required anymore by the evaluation algorithm, detecting that it will not be reached from any newly added node, thus allowing to delete the node and all the nodes reachable from it. To implement garbage collection, the $\ds$ module keeps a pointer count for each node in the memory pool. Once the pointer counter hits zero, it moves the node to a pool of nodes that can be deleted. As stated previously, the memory locations of these nodes are not liberated but are instead overwritten by newly created nodes, in which case these locations are removed from the pool of nodes to be deleted.

%%REWRITTEN
\section{Node manager (Optimization)}

The key step when evaluating a REQL query over a document is building the ECS which encodes all the outputs. Naively building the ECS would require allocating the memory for each node being added, which is highly inefficient. Therefore, \rematch\ includes a \emph{node manager} module, denoted $\ds$, which allocates memory in bulk, and acts as a garbage collector for the ECS. The $\ds$ module essentially creates a memory pool for storing the nodes in the ECS. Each time $\ds$ fills the pool, it allocates the memory for another pool double the original size. This strategy allows memory allocation to occur infrequently and in big chunks, thus preventing fragmentation and multiple pointer dereferencing. In addition, the $\ds$ module acts as a lazy garbage collector by keeping a pointer count for each node in the memory pool. Once the pointer counter hits zero, it moves the node to a pool of nodes that can be deleted. This will happen when a node is not required anymore by the evaluation algorithm, detecting that it will not be reached from any newly added node, thus allowing to delete the node and other nodes whose pointer count hits zero by this removal.
%all the nodes reachable from it.
%\domagoj{Strictly speaking, some reachable nodes might be used by another run. Perhaps we should just say that the unused node is deleted and the ref count is updated for reachable nodes?}
 As stated previously, the memory locations of deleted nodes are not liberated but are instead overwritten by newly created nodes, in which case these locations are removed from the pool of nodes to be deleted. We use the command $\ds.\texttt{discard}(\coid)$ when we wish to signal that a node is to be discarded.


The node manager $\ds$ is also in charge of implementing the following three operations over nodes:
\begin{itemize}
	\item $\ds.\cinit$, which creates the terminal node;
	\item $\ds.\cextend(\coid,(S,i))$, which creates a content node which contains $(S,i)$, and links this newly created node to $\coid$; and
	\item $\ds.\cunion(\coid_1,\coid_2)$, which takes nodes $\coid_1$ and $\coid_2$, and creates a new union node, representing the union of both outputs.
\end{itemize}
$\ds$ implements these three operations, taking constant time for each operation. Moreover, $\ds$ module supports enumerating outputs from any given node $\coid$, which we denote by $\ds.\cenumerate(\coid)$. More importantly, this enumeration takes output-linear delay, and one can do it at any point without further preprocessing. This last fact is the guarantee for \rematch\ to retrieve all outputs $\sem{\texttt{e}}_d$ with output-linear delay. We refer to~\cite{MunozR22} for the implementation details of such operations and enumeration procedure. 



%While the first two operation can be implemented in a straightforward fashion, asargue, $\cunion$ is a bit more involved, as it needs to ensure that no output enumerated from $n_1$ is repeated when enumerating from $n_2$, and vice versa. To achieve this, a special construction is deployed which we illustrate next.
%
%The $\ds$ module supports the three operations for implementing the ECS, as explained above. When manipulating the ECS in \rematch, we will therefore write $\ds.\cextend(n,(C,i))$ to denote that the node $n$ is pointed to by a newly created node with the content $(C,i)$. Finally, the $\ds$ module supports enumerating outputs from a given output node, which we denote by $\ds.\cenumerate(n)$.


\begin{figure}[t]
	\begin{tikzpicture}[->,>=stealth, semithick, auto, initial text= {}, initial distance= {3mm}, accepting distance= {4mm}, node distance=1.2cm, semithick]
		\tikzstyle{every state}=[draw=black,text=black,inner sep=0pt, minimum size=10mm]
		
		
		\node [state] (15) at (0,0) {$\bot$};
		
		\node [state] (11) at ($(15)+(2,0.8)$) {$[x,0$};
		\node [state] (10) at ($(11)+(1.8,0)$) {$x\rangle, 2$};
		
		\node [state] (14) at ($(15)+(2,-0.8)$) {$[x, 3$};
		\node [state] (9) at ($(14)+(1.8,0)$) {$x\rangle,5$};
		
		\node [state] (7) at ($(9)+(1.8,0)$) {$\cup$};
		
		\node [state] (5) at ($(10)+(3.6,0)$) {$[y, 4$};
		\node [state] (6) at ($(7)+(1.8,0)$) {$[y, 7$};
		
		\node [state] (3) at ($(5)+(1.8,0)$) {$y\rangle, 7$};
		\node [state] (4) at ($(6)+(1.8,0)$) {$y\rangle,\!10$};
		
		\node [state] (2) at ($(3)+(1.8,-0.8)$) {$\cup$};
		
		\draw [->] (2) to (3);
		\draw [->] (2) to (4);
		\draw [->] (3) to (5);
		\draw [->] (4) to (6);
		\draw [->] (7) to (9);
		\draw [->] (6) to (7);
		\draw [->] (10) to (11);
		\draw [->] (5) to (10);
		\draw [->] (7) to (10);
		\draw [->] (9) to (14);
		\draw [->] (11) to (15);
		\draw [->] (14) to (15);
	\end{tikzpicture}
	\caption{The ECS representing three outputs.}
	\label{fig-ecs}
	\vspace{-4mm}
\end{figure}




\section{Early output (Optimization)} 

Note again that $\ds$ allows enumerating the outputs from a node at any point without further preprocessing. This fact is crucial for the optimization that we call \emph{early output enumeration}. Specifically, when evaluating a logical VA over a document, we can often detect whether we reach a final state before the entire document is read completely (for details, see Chapter~\ref{chpt:evaluation}). In essence, this means that we can provide certain outputs to the user at this point before continuing to construct the ECS in its entirety. The benefits are twofold: (i) the outputs are delivered to the user as soon as possible, similarly as pipelined evaluation is done in databases; and (ii) once we enumerate these outputs, we can delete the unused nodes, thus saving storage space. This optimization is highly effective in decreasing the memory usage and decreasing the time to deliver the first output (see Section~\ref{sec:experiments}). %, as shown experimentally in Section~\ref{sec:experiments}.

 %For instance, in the ECS from Figure~\ref{fig-ecs} the node $(y\rangle,7)$ can be detected as an output node, so the nodes $(y\rangle,7)$ and $([y,4)$ can be discarded after enumerating this output. Notice that the nodes $([x,0)$ and $(x\rangle,2)$ can not be eliminated since they are still part of another output.



%-------------------
%
%We discuss the output structure, enumeration, and garbage collector:
%\begin{enumerate}
%	\item Data structure from Martin. 
%	\item Enumeration. 
%	\item Garbage collector.
%	\item Early output.
%\end{enumerate}
%\cristian{I left early output here, because the next section looks to big, and, if we can explain it here, it makes more sense.}
%
%\cristian{Define here the connection between output mappings and output sequences. The idea is to represent a mapping as a sequence of pairs (set of markers, index). The idea would be to explain the representation here with the data structure. To then use it in the next section. }



%\paragraph*{Enumeration with output-linear delay.} 
%We wish to be careful with what we mean by ``efficiently generate''
%$\tab(M)$ from $R$ when $M$ is finite. Indeed, because $R$ can be exponentially
%more succinct than $M$ and $\tab(M)$, the total time to generate $\tab(M)$ from
%$R$ will obviously be exponential in $R$ in the worst case. This exponential
%complexity is only due to the exponential number of tuples that we need to
%generate: we will show that generating individual tuples in $\tab(M)$ from $R$
%is efficient, in the sense that it takes only time proportional to the size of the tuple
%being generated---independently of the size of $R$, or $M$, or $\tab(M)$. To
%formalize this notion, we adopt the framework of \emph{enumeration
%  algorithms}. Enumeration algorithms are an attractive way of gauging the
%complexity of algorithms that need to generate large (or infinite) sets, which
%have recently received significant attention in the database
%community, both from a theoretical~\cite{AmarilliBMN-tods21,FlorenzanoRUVV20,LosemannM-lics14,SchweikardtSV-jacm22,BerkholzGS-siglog20,Segoufin-icdt13} and practical point~\cite{IdrisUVVL-vldbj20,IdrisUV-sigmod17,TziavelisAGRY-pvldb20}.
%
%We require the following definitions. Given an input $x$, an algorithm is said
%to \emph{enumerate} a multiset $O$ if it outputs the elements of $O$ one by one
%in some order $o_1, o_2, o_3, \dots$, such that the number of times an element $o$ occurs in this enumeration equals its multiplicity in $O$. (Repeated elements need not be subsequent in the enumeration.) In particular, if $O$ is a set, then the enumeration cannot contain duplicates. It enumerates $O$
%\emph{with output-linear delay} if the time required to output the $i$-th element $o_i$,
%measured as the difference in time between outputting $o_{i-1}$ (or the start of
%the algorithm, when $i=1$) and finishing outputting $o_i$, is proportional to
%the size of $o_i$, independent of the size of $O$ or of the input $x$. If $O$ is
%finite, then it is also required that the algorithm terminates immediately after
%outputting the last element.  In that case, the total time that the algorithm
%takes to enumerate $O$ is hence $\bigo(\size{O})$, i.e., linear in $O$.