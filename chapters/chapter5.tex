% !TeX spellcheck = en_US !TEX root = ../Thesis.tex

To evaluate REQL, we compile the logical VA into a so-called \emph{extended VA},
a ``physical'' automata model closer to the evaluation algorithm than to the
query language. We use this model for the REQL evaluation algorithm used in
\rematch, which makes one pass over the document and produces the resulting
mappings with output-linear delay. This evaluation algorithm refines and
improves the theoretical algorithm presented in \citet{FlorenzanoRUVV20} for
evaluating extended VA by taking care of the memory usage, simplifying the
algorithm, and applying various optimizations presented in the previous
sections.

\section{Extended VA}

An \emph{extended variable-set automaton} (\(\EVA\))~\citep{FaginKRV15} is a
finite-state automaton extended with capture variables in a way analogous to
logical VAs. The difference is that reading and outputting variable markers can
be done in a single transition\footnote{The name $\EVA$ is also used
by~\citet{FlorenzanoRUVV20} for an automaton model outputting sets of markers.
However, the eVA of~\citet{FlorenzanoRUVV20} require letter and variable
transitions to be separate, while we process them simultaneously. We use the
same name for simplicity.}
%This extension will considerably simplify the algorithm and justify the use of
%a new model, more closed to the evaluation strategy than to the REQL query.
Formally, an $\EVA$ $\cE$ is a tuple \((Q, q_0, F, \delta)\), where \(Q\) is a
finite set of \textit{states}; $q_0 \in Q$ is the initial state; $F\subseteq Q$
is the set of final states; and $\delta$ is a \textit{transition relation}
consisting of transitions of the form $(q, a, S, q')$ where $q, q' \in Q$, $a
\in \Sigma \cup \{\osymbol\}$, and $S$ is a set of variable markers (e.g.,
$S=\{\Open{x},\Close{y}\}$). The meaning behind the transition $(q, a, S, q')$
is that when the automaton is in the state $q$ and reads the $i$-th letter $a_i
= a$, it switches to state $q'$ and outputs $(S, i)$, where $S$ is the set of
variables that are opened and closed \emph{before} reading the $i$-th letter.
Intuitively, in a logical VA one could have a run with many contiguous variable
transitions, followed by a letter transition. In an eVA, this would translate to
a single transition, where $S$ holds all the variable markers from the
contiguous variable transitions, and $a$ is the letter from the letter
transition. In particular, when $S = \emptyset$, the automaton changes the state
and nothing is output.

%$\osymbol$ is used to simulate opening variables before reading the first
%letter of the document. We define the set $\Var(\cE)$ as the set of all
%variables $x$ that are mentioned in some transition of $\cE$.

The fact that a transition can open or close multiple variables at the same time
allows us to handle nested variables (e.g. \texttt{!x\{!y\{a\}\}}) in a single
automaton transition. Furthermore, the $\osymbol$ symbol is used as a special
input to denote the end of a document (i.e., an End Of File, or \texttt{EOF}).
This will be useful in the run of an $\EVA$ for outputting spans $[i, j\rangle$
over a document of size $n$, where $i$ and $j$ range from $0$ to $n$ (i.e., we
need $n+1$ possible positions).

Next we define how an $\EVA$ produces outputs. A \emph{run} $\rho$ of $\cE$ over
a document $d = a_0 \cdots a_{n-1}$ is a sequence of the form:
\begin{equation}\label{eq:run}
	\rho \ = \ q_0 \ \longtrans{b_0 / S_0} \ q_1 \ \longtrans{b_1 / S_1} \ q_2 \ \longtrans{b_2 / S_2} \ \ldots \ \ q_{n} \longtrans{b_{n} / S_n} \ q_{n+1}
\end{equation}
where $b_0 b_1 \ldots b_n = a_0 \ldots a_{n-1} \osymbol$ (the document with an
special symbol $\osymbol$), $S_j$ is a set of variable markers, and
$(q_j,b_{j},S_{j}, q_{j+1}) \in \delta$, for $0\leq j\leq n$. Also, we say that
$\rho$ is \emph{accepting} if $q_{n+1} \in F$. A run $\rho$ like (\ref{eq:run})
naturally defines an output sequence as $\oout(\rho) = \oout(S_0, 0) \cdot
\ldots \cdot \oout(S_n, n)$ such that $\oout(S_i, i) = (S_i, i)$ if $S_i \neq
\emptyset$, and $\epsilon$, otherwise. Finally, the semantics of $\cE$ over $d$,
denoted by $\semseq{\cE}_d$, is defined as the set of all output sequences
$\oout(\rho)$ where $\rho$ is an accepting run of $\cE$ over $d$.

$\EVA$s have several differences compared to logical VAs: (1) an $\EVA$ starts
from the beginning of the document, (2) produces output sequences instead of
mappings, and (3) the transitions are fired by symbols (i.e., not by char
classes). These relaxations plus read-captures transitions will considerably
simplify the evaluation algorithm.  Indeed, we can show that if we start from a
logical VA $\cA$, then we can construct an equivalent $\EVA$ $\cE$ in linear
time. %, for evaluating $\cA$.
\begin{proposition}\label{prop:eVAconstruction}
	%	For every logical VA $\cA$ compiled from a REQL query, one can build in
	%	linear time an $\EVA$ $\cE$ such that $\cE$ evaluated over a document
	%	$d$ produces the same output sequences of mappings from $\cA$ evaluated
	%	over $d$.
	For every logical VA $\cA$ compiled from a REQL query \texttt{e} and
	document $d$, one can build, in linear time, an $\EVA$ $\cE$ such that $\cE$
	evaluated over $d$ produces precisely the output sequences representing the
	mappings in $\sem{\cA}_d = \sem{\texttt{e}}_d$.
\end{proposition}

\begin{proof}
	To construct an equivalent extended Variable Automaton (eVA) $\cE$ from a
	document $d$ and a logical VA $\cA = (Q, \delta, q_0, q_f)$ derived from a
	REQL query \texttt{e}, one can adopt a procedure akin to that for
	$\varepsilon$-logical VAs from the proof of Proposition
	\ref{prop:logicalVA}. This process begins by defining, for each state $q \in
	Q$, a \emph{variable-set closure}, $\vclosure(q)$. This closure encompasses
	all states that are accessible from $q$ solely through variable transitions.
	\cristian{``through \textbf{zero or more} variable transitions''.}
	For any state $r$ within $\vclosure(q)$, a series of variable transitions
	must exist: $(r_0, o_0, r_1), \ldots, (r_n, o_n, r_{n+1})$, where $r_0 = q$,
	$r_{n+1} = r$, and each $o_i$ represents a variable marker for $0 \leq i
	\leq n$. 
	 \cristian{Yo cambiaria la linea anterior por: For any state $r$ within $\vclosure(q)$, a sequence (possibly empty) of variable transitions
	 	must exist: $(r_0, o_1, r_1), (r_1, o_2, r_2), \ldots, (r_{n-1}, o_n, r_{n})$, where $r_0 = q$,
	 	$r_{n} = r$, and each $o_i$ represents a variable marker for $0 \leq i
	 	\leq n$. }
	
	Given $\cA$'s basis in a REQL query, 
	\cristian{``Given $\cA$'s basis in a REQL query \textbf{that is well-design},''. Notar que lo de well-design es lo que hace la diferencia} 
	the variable markers set
	$\{o_i \mid 0 \leq i \leq n\}$ remains consistent across all sequences
	leading to $r$. For each transition $(r, C, s) \in \delta$, where $C$ is a
	character class, a new transition $(q, a, \{o_i \mid 0 \leq i \leq n\}, s)$
	is added to the updated transition relation $\delta'$ if and only if $a \in
	C$ and appears in $d$.
	\cristian{Aqui agregaria el cambio de que ExtendedVA puede ocupar charclasses.}
	Furthermore, if $r$ equals $q_f$, then $q$ should be
	included in $\cE$'s new accepting states set $F$. 
	\cristian{Furthermore, if $r$ equals $q_f$, then we add a transition $(q, \osymbol, \{o_i \mid 0 \leq i \leq n\}, q_f)$.}
	Lastly, for every
	transition of the form $(q, C, q')$, add $(q, a, \emptyset, q')$ to
	$\delta'$ iff $a \in C$ and appears in $d$. 
	\cristian{Aqui también agregas el cambio de que ExtendedVA puede ocupar charclasses.}
	The resulting eVA $\cE = (Q,
	\delta', q_0, F)$ will satisfy $\sem{\cE}_d = \sem{\cA}_d$.
\end{proof}


\begin{example}\label{ex:lva-eva} To illustrate
	Proposition~\ref{prop:eVAconstruction}, recall query $\texttt{e0}$ and its
	equivalent logical VA $\cA_0$ from Example~\ref{example:logicalVA}. The
	following extended VA $\cE_0$ is equivalent to $\cA_0$:
	%	\begin{center} \begin{tikzpicture}[->,>=stealth, semithick, auto,
	%	    initial text= {}, initial distance= {3mm}, accepting distance=
	%	    {4mm}, node distance=1.13cm, semithick] \tikzstyle{every
	%	    state}=[draw=black,text=black,inner sep=0pt, minimum size=4.3mm]
	%	    \node[initial below,state] (0) {$0$}; \node[state] (2) [right of=0]
	%	    {$2$}; \node[state] (3) [right of=2] {$3$}; \node[state] (4) [right
	%	    of=3] {$4$}; \node[state] (6) [right of=4] {$6$}; \node[state] (7)
	%	    [right of=6] {$7$}; \node[state] (8) [right of=7] {$8$};
	%	    \node[state,accepting] (9) [right of=8] {$9¨$};
	%
	%	        \draw[->] (0) edge[loop above] node {$*/\emptyset$} (0);
	%	        \draw[->] (0) edge node {$t/\Open{x}$} (2);
	%	        \draw[->] (2) edge node {$h/\emptyset$} (3);
	%	        \draw[->] (3) edge node {$* / \Close{x}$} (4);
	%	        \draw[->] (4) edge[loop above] node {$*/\emptyset$} (4);
	%	        \draw[->] (3) edge[bend right,swap] node {$h \, / \, \{\, \Close{x}, \Open{y}\, \}$} (6);
	%	        \draw[->] (4) edge node {$h/\Open{y}$} (6);
	%	        \draw[->] (6) edge node {$a/\emptyset$} (7);
	%	        \draw[->] (7) edge node {$t/\emptyset$} (8);
	%	        \draw[->] (8) edge node {$*/\Close{y}$} (9);    
	%	        \draw[->] (9) edge[loop above] node {$*/\emptyset$} (9);
	%	    \end{tikzpicture}
	%	\end{center}

	\begin{center}
		\begin{tikzpicture}[->,>=stealth, semithick, auto, initial text= {}, initial distance= {3mm}, accepting distance= {4mm}, node distance=1.9cm, semithick]
			\tikzstyle{every state}=[draw=black,text=black,inner sep=0pt,
			minimum size=7mm]

			\begin{scope}
				\node[initial,state] (0) {$0$}; \node[state] (2) [right of=0]
				{$1$}; \node[state] (3) [right of=2] {$2$}; \node[state] (4)
				[right of=3] {$3$}; \node[state] (5) [right of=4] {$4$};
				\node[state,accepting] (6) [right of=5] {$5$};

				\draw[->] (0) edge[loop above] node {$*/\emptyset$} (0);
				\draw[->] (0) edge node {$t/\{\Open{x}\}$} (2); \draw[->] (2)
				edge node {$h/\emptyset$} (3); \draw[->] (3) edge node
				{$a/\emptyset$} (4); \draw[->] (4) edge node {$t/\emptyset$}
				(5); \draw[->] (5) edge node {$*/\{\Close{x}\}$} (6); \draw[->]
				(6) edge[loop above] node {$*/\emptyset$} (6);
			\end{scope}
		\end{tikzpicture}
	\end{center}
	Here we draw an edge $q \ \longtrans{a / S} \ q'$ to denote a transition
	from $q$ to~$q'$ that outputs $S$ when reading a letter $a$, and use $*$ to
	denote any letter. Intuitively, we have moved transitions with variable
	markers ``forward'' if we compare $\cE_0$ with $\cA_0$.  For instance,
	$\cA_0$ had transitions $0 \trans{\Open{x}} 1 \trans{t} 2$, but instead
	$\cE_0$ has a direct transition $0 \longtrans{t/\Open{x}} 1$.  Since an
	extended VA needs to consume the entire document, we add self-loops in the
	initial and the final state; for instance $4 \longtrans{t/\Close{x}} 5$.
	Also note that the transition $0 \longtrans{*/\emptyset} 0$ closes the
	variable \texttt{x} before the character (\textsf{EOF} or any other) is
	consumed.
\end{example}

\section{Determinization}
There is a critical issue with using $\EVA$ as our guide for evaluating REQL:
several runs can produce the same output sequence. To illustrate this issue in
its simplest form, assume the following extended VA $\cE_0'$, which is a mild
modification of $\cE_0$:
\begin{center}
	\begin{tikzpicture}[->,>=stealth, semithick, auto, initial text= {}, initial distance= {3mm}, accepting distance= {4mm}, node distance=2.2cm, semithick]
		\tikzstyle{every state}=[draw=black,text=black,inner sep=0pt, minimum
		size=7mm]
		\begin{scope}
			\node[initial,state] (0) {$0$}; \node[state] (2) [right of=0] {$1$};
			\node[state] (3) [right of=2] {$2$}; \node[state] (4) at
			($(3)+(1.7,0.6)$)  {$3$}; \node[state] (4') at ($(3)+(1.7,-0.6)$)
			{$3'$}; \node[state] (5) at ($(4)+(1.7,-0.6)$)  {$4$};
			\node[state,accepting] (6) [right of=5] {$5$};

			\draw[->] (0) edge[loop above] node {$*/\emptyset$} (0); \draw[->]
			(0) edge node {$t/\{\Open{x}\}$} (2); \draw[->] (2) edge node
			{$h/\emptyset$} (3); \draw[->] (3) edge node[pos=0.7]
			{$a/\emptyset$} (4); \draw[->] (4) edge node[pos=0.3]
			{$t/\emptyset$} (5); \draw[->] (3) edge node[swap,pos=0.7]
			{$a/\emptyset$} (4'); \draw[->] (4') edge node[swap,pos=0.3]
			{$t/\emptyset$} (5); \draw[->] (5) edge node {$*/\{\Close{x}\}$}
			(6); \draw[->] (6) edge[loop above] node {$*/\emptyset$} (6);
		\end{scope}
	\end{tikzpicture}
\end{center}
One can check that $\cE_0'$ is equivalent to $\cE_0$, namely,
$\semseq{\cE_0'}(d) = \semseq{\cE_0}(d)$ for every document $d$. However, for
every output sequence of $\semseq{\cE_0'}(d)$, two different runs are witnessing
it: one crossing the state $3$ and another crossing the state $3'$. Then, if we
guide an evaluation algorithm of eVA with runs, for $\cE_0'$ we will enumerate
each output sequence twice, although the user expects to extract each output
without duplicates. Similar behavior can happen for eVAs coming from REQL
queries with quantifiers or alternations.

To remove duplicate runs from $\EVA$s, we use a subclass called
\emph{deterministic} $\EVA$s. We say that an $\EVA$ $\cE$ is
\emph{deterministic} if its transition relation $\delta$ satisfies that for
every two transitions $(q, a_1, S_1, q_1') \in \delta$ and $(q, a_2, S_2, q_2')
\in \delta$, if $(a_1, S_1) = (a_2, S_2)$, then $q_1' = q_2'$. In other words,
given a state $q$, the next state is determined by the pair $(a, S)$. The reader
can check that $\cE_0$ is deterministic but $\cE_0'$ is not.
\begin{proposition} \label{prop:determinization} Let $\cE^{\odet}$ be a
	deterministic $\EVA$. Then, for every document $d$ and output sequence
	$(S_1, i_1)\cdot \cdots \cdot(S_n, i_n)$, there is at most one accepting run
	$\rho$ of $\cE^{\odet}$ over $d$ such that $\oout(\rho) = (S_1, i_1)\cdot
	\cdots \cdot(S_n, i_n)$.
\end{proposition}
\begin{proof}
	Consider $\cE^{\odet}$, a deterministic eVA. Let $(S_1, i_1)\cdot \ldots
	\cdot (S_m, i_m)$ be an output sequence and $d = a_0\ldots a_n$ a document.
	Assume that $(S_1, i_1)\cdot \ldots \cdot (S_n, i_n)$ is in $\semseq{\cE}_d$
	(the case where it is not is trivial). Under this assumption, there must
	exist at least one accepting run $\rho$ of $\cE^{\odet}$ over $d$:
$$
\rho \ = \ q_0 \ \longtrans{b_0 / R_0} \ q_1 \ \longtrans{b_1 / R_1} \ q_2 \ \longtrans{b_2 / R_2} \ \ldots \ \ q_{n} \longtrans{b_{m} / R_n} \ q_{n+1}
$$
such that $\oout(\rho) = (S_1, i_1)\cdot \cdots \cdot (S_m, i_m)$. Now, suppose
there is another accepting run $\rho'$ of $\cE^{\odet}$:
$$
\rho' \ = \ p_0 \ \longtrans{b_0 / T_0} \ q_1 \ \longtrans{b_1 / T_1} \ p_2 \ \longtrans{b_2 / T_2} \ \ldots \ \ p_{\ell} \longtrans{b_n / T_n} \ p_{n+1}
$$
over $d$, where $\rho' \neq \rho$ but $\oout(\rho') = \oout(\rho)$. First,
observe that
\begin{equation} \label{eq:outseq}
\oout(R_0, 0)\cdot \ldots \cdot \oout(R_n, n) = \oout(T_0, 0)\cdot \ldots \cdot \oout(T_n, n) = (S_1, i_1)\cdot \ldots \cdot (S_m, i_m).
\end{equation}
This implies that $R_j = T_j$ for all $0 \leq j \leq n$. If there were some $j$
where $R_j \neq T_j$, then $\oout(R_j, j) \neq \oout(T_j, j)$ would hold. This
would mean that $R_j$ and $T_j$ cannot both be empty simultaneously, and thus,
one, and only one of $(R_j, j)$ or $(T_j, j)$ must be present in the output
sequence. Let us assume that $(R_j, j)$ is present in the output sequence, then
$R_j = S_\ell$ with $1 \leq \ell \leq m$ satisfying that $i_\ell = j$. However,
this contradicts equation \eqref{eq:outseq} since $(S_\ell, i_\ell) = (R_j, j)
\neq \oout(T_j, j)$, which implies necessarily that $\oout(T_0, 0)\cdot \ldots \cdot
\oout(T_n, n) \neq (S_1, i_1) \cdot \ldots \cdot  (S_m, i_m)$.

Next, considering that $\rho'$ and $\rho$ are distinct and $q_0 = p_0$, there
must exist a smallest index $0 \leq j \leq m$ for which $q_j = p_j$ and $q_{j+1}
\neq p_{j+1}$. This implies that both $(q_j, b_j, R_j, q_{j+1})$ and $(q_j, b_j,
R_j, p_{j+1})$ are transitions in $\cE^\odet$. However, this is a contradiction
because $\cE^{\odet}$ is deterministic, and hence it cannot have two different
transitions from the same state with the same input and output. Therefore, we
must conclude that $\rho' = \rho$, establishing the uniqueness of $\rho$.
\end{proof}

Proposition \ref*{prop:determinization} is crucial for our evaluation algorithm,
given that we can simulate runs and construct the output, without worrying about
duplicates. Fortunately, we can always ``determinize'' a non-deterministic
$\EVA$ $\cE = (Q, q_0, F, \delta)$ by using a subset construction, similar to
the standard determinization procedure for NFAs. More precisely, we define
$\cE^{\odet} = (Q^{\odet}, q_0^{\odet}, F^{\odet}, \delta^{\odet})$ such that:
$Q^{\odet} = 2^Q$, $q_0^{\odet} = \{q_0\}$, $F^{\odet} = \{X \mid X \cap F \neq
\emptyset\}$, and:
$$
	\delta^{\odet} = \left\{(X, a, S, X') \mid \forall q' \in Q.\ q' \in X' \, \rightarrow \,  \left(\exists q \in X.\ (q, a, S, q') \in \delta\right)\right\}.
$$

\begin{proposition}\label{prop:eVAdet} For every $\EVA$ $\cE$ and document $d$,
	the $\EVA$ $\cE^{\odet}$ is equivalent to $\cE$, namely,
	$\semseq{\cE^{\odet}}_d = \semseq{\cE}_d$
\end{proposition}

\begin{proof}
	Let $d=a_0\ldots a_{n-1}$, $\cE = (Q, q_0, F, \delta)$ and $\cE^{\odet} =
		(Q^{\odet},q_0^{\odet}, F^{\odet}, \delta^{\odet})$.

	Let $(S_1, i_1)\cdot \cdots \cdot(S_m, i_m) \in \semseq{\cE}_d$ with $0 \leq
		m \leq n$. Then, there is an accepting run $\rho$ of $\cE$ over $d$:
	$$
		\rho \ = \ q_0 \ \longtrans{a_0 / R_0} \ q_1 \ \longtrans{a_1 / R_1}  \ldots \ q_{n-1} \longtrans{a_{n-1} / R_{n-1}} \ q_{n} \longtrans{\osymbol / R_n} \ q_{n+1}
	$$
	such that $\oout(\rho) = (S_1, i_1)\cdot \cdots \cdot(S_n, i_n)$. There must
	also exist a run $\rho^{\odet}$ of $\cE^{\odet}$ over $d$:
	$$
		\rho^{\odet} \ = \ X_0 \ \longtrans{a_0 / R_0} \ X_1 \ \longtrans{a_1 / R_1}  \ldots \ X_{n-1} \longtrans{a_{n-1} / R_{n-1}} \ X_{n} \longtrans{\osymbol / R_n} \ X_{n+1}
	$$
	with $X_0 = q_0^{\odet}$. We claim that $q_i \in X_i$ for $0 \leq i \leq n +
		1$, and thus, $\rho^{\odet}$ is accepting. By induction, it is trivial
		that $q_0 \in X_0$ (base case). Then, if $q_i \in X_i$, as $(q_i, a_i,
		S_i, q_{i+1}) \in \delta$ and $(X_i, a_i, R_i, X_{i+1}) \in
		\delta^{\odet}$, it must be true that $q_{i+1} \in X_{i+1}$ (inductive
		hypothesis). Given that $q_{n+1} \in F$, we have that $X_{n+1} \in
		F^{\odet}$, and thus, $\rho^{\odet}$ is accepting. Thus,
		$\oout(\rho^{\odet}) = (S_1, i_1)\cdot \cdots \cdot(S_n, i_n)$, and
		hence, $(S_1, i_1)\cdot \cdots \cdot(S_n, i_n) \in
		\semseq{\cE^{\odet}}_d$.

	Conversely, let $(S_1, i_1)\cdot \cdots \cdot(S_m, i_m) \in
		\semseq{\cE^{\odet}}_d$ with $0 \leq m \leq n$. Then, there is an
		accepting run $\rho^{\odet}$ of $\cE^{\odet}$ over $d$:
	$$
		\rho^{\odet} \ = \ X_0 \ \longtrans{a_0 / R_0} \ X_1 \ \longtrans{a_1 / R_1}  \ldots \ X_{n-1} \longtrans{a_{n-1} / R_{n-1}} \ X_{n} \longtrans{\osymbol / R_n} \ X_{n+1}
	$$
	such that $\oout(\rho^{\odet}) = (S_1, i_1)\cdot \cdots \cdot(S_n, i_n)$. We
	claim that for each $i \leq n$ and $q \in X_i$, there exists a run:
	$$
		\rho_{q,i} \ = \ q_0 \ \longtrans{a_0 / R_0} \ q_1 \ \longtrans{a_1 / R_1}  \ldots \longtrans{a_{i} / R_{i}} \ q_{i} = q.
	$$
	of $\cE$ over $d$. By induction, if the claim is true for $i$, then as
	$(X_i, a_{i+1}, S_{i+1}, X_{i+1}) \in \delta^{\odet}$ with $X_{i+1} = \{q'
	\in Q \mid \exists q \in X_i.\ (q, a, S, q') \in \delta\}$, then there must
	exist $q' \in X_{i}$ such that $(q', a_{i+1}, S_{i+1}, q) \in \delta$. Thus, 
	$$
		\rho_{q,i} \ = \ q_0 \ \longtrans{a_0 / R_0} \ q_1 \ \longtrans{a_1 / R_1}  \ldots \longtrans{a_{i} / R_{i}} \ q_{i} = q' \ \longtrans{a_{i+1} / R_{i+1}} \ q_{i+1} = q.
	$$
	must exist as a run of $\cE$ over $d$. By induction, then the claim is true
	for $q \in X_{n+1}$, and as $X_{n+1} \cap F \neq \emptyset$, we take $q \in
	X_{n+1} \cap F$. Thus, $\rho_{q, n}$ is an accepting run of $\cE$ over $d$,
	and $\oout(\rho_{q, n}) = (S_1, i_1)\cdot \cdots \cdot(S_n, i_n)$.
	Therefore, $(S_1, i_1)\cdot \cdots \cdot(S_n, i_n) \in \semseq{\cE}_d$.
\end{proof}


An inconvenience of $\cE^{\odet}$ is that its size can be exponential in $|\cE|$
\citep{HopcroftU79}. Fortunately, for the evaluation algorithm it is not
necessary to construct % TODO: reference
entire $\cE^{\odet}$. Instead, we can start from the initial set $\{q_0\}$ and
traverse only the transitions and states needed by the next letter. Each time
that we need a new state or transitions of $\cE^{\odet}$, we use $\cE$ to build
them and cache it in main memory for future access. This is the purpose of the
\emph{determinization module} of \rematch, called $\cdet$. Specifically, $\cdet$
has a method $\mdetnext$ such that, given a state $X \in Q^\odet$ and a letter
$a$, $\cdet.\mdetnext(X, a)$ computes a list $\ell$ with all pairs $(S, X')$
such that $(X,a,S,X') \in \delta^\odet$. Instead, if $\ell$ was already computed
before (and cached), $\cdet.\mdetnext(X, a)$ outputs $\ell$ immediately. Then by
using $\cdet$ we only need to compute each state and transition once. Moreover,
the number of states accessed by $\cdet$ depends on $\cE$ and the input document
$d$. In practice, this size is small and at most three or four times the size of
$\cE$.

\begin{algorithm*}
	\caption{Evaluation of an extended variable-set automaton $\cE =
			(Q,\delta,q_0,F)$ over the document $b_0 \ldots b_{n}$ where $b_0
			\ldots b_{n-1}$ is the original document and $b_n = \osymbol$ is an
			\texttt{EOF} symbol. $\cdet$ and $\ds$ are the determinization and
			node manager modules, respectively.}\label{alg:evaluation}
	\smallskip
	\begin{varwidth}[t]{0.5\textwidth}
		\begin{algorithmic}[1]
			\Procedure{Evaluate}{$\cE, b_0 \ldots b_{n}$} \State
			$\cdet.\texttt{initialize}(\cE)$ \State $\textproc{InitializeLists}$
			\For{$i=0$ \textbf{to} $n$} \ForAll{$X \in \cstateslists$} \State
			$\ell \gets \cdet.\texttt{next}(X,b_i)$ \If{$\ell \neq
			\texttt{empty}$} \State $\textproc{UpdateSets}(X,\ell,i)$ \Else
			\State $\ds.\texttt{garbage}(X.\coid)$ \EndIf \EndFor \State
			$\cstateslists.\texttt{swap}(\cstateslists')$ \State
			$\cstateslists'.\texttt{clear}$ \EndFor \State
			$\textproc{Enumerate}$ \EndProcedure \State \State
			\Procedure{Enumerate}{} \ForAll{$X \in \cstateslists$}
			\If{$X.\texttt{isFinal}$} \State $\ds.\cenumerate(X.\coid)$ \EndIf
			\EndFor \EndProcedure
			\algstore{myalg}
		\end{algorithmic}
	\end{varwidth} \hfill
	\begin{varwidth}[t]{0.5\textwidth}
		\begin{algorithmic}[1]
			\algrestore{myalg}
			\Procedure{InitializeLists}{} \State $\cstateslists.\texttt{clear}$
			\State $\cstateslists'.\texttt{clear}$ \State $X_0 \gets
			\cdet.\texttt{initialStateSet}$ \State $X_0.\citer \gets -1$ \State
			$X_0.\coid \gets \ds.\cinit$ \State
			$\cstateslists.\texttt{add}(X_0)$ \EndProcedure \State
			\Procedure{UpdateSets}{$X,\ell,i$} \ForAll{$(S,X') \in \ell$} \State
			$\coid' \gets X.\coid$ \If{$S \neq \emptyset$} \State $\coid' \gets
			\ds.\cextend(\coid', (S, i))$ \EndIf \If{$X'.\citer < i$} \State
			$X'.\citer \gets i$ \State $\cstateslists'.\texttt{add}(X')$ \State
			$X'.\coid \gets \coid'$ \Else \State $X'.\coid \gets
			\ds.\cunion(X'.\coid, \coid')$ \EndIf \EndFor \EndProcedure
		\end{algorithmic}
	\end{varwidth}
\end{algorithm*}

\section{Algorithm's variables}
In Algorithm~\ref{alg:evaluation} we present the main algorithm for \rematch.
This algorithm evaluates an $\EVA$ $\cE$ over the input document $d$,
enumerating all output sequences $\semseq{\cE}(d)$. %Next, we introduce the
variables and data structures used by the algorithm.
%
Two main components used by the algorithm are the node manager $\ds$, introduced
in Chapter~\ref{chpt:output}, and the determinization module $\cdet$, introduced
above. In addition, we use \emph{state-sets} constructed by $\cdet$,
\emph{set-lists} that store state-sets, and \emph{nodes} created and operated by
$\ds$. We introduced nodes $\coid$ in Chapter~\ref{chpt:output}. %Next, we
explain how state-sets and set-lists work.

% TODO: Difference between evas and transducers

The state-sets, denoted by $X$ in the algorithm, are built and cached by the
determinization module for representing a set $X \subseteq Q$. Each state-set
$X$ has two variables: $X.\coid$ and $X.\citer$. The former can store a
reference to an ECS node that represents the current outputs of runs that
reached $X$ (recall that every node $\texttt{n}$ in ECS encodes a set of
mappings $\sem{n}$). Instead, the latter is an integer that encodes the current
phase number: if $X.\citer = i$ then the $i$-th iteration was the last one that
reached $X$. In practice, $\citer$ will help to know whether it is the first
time we reached $X$ during some iteration. Further, each state-set has a method
$X.\texttt{isFinal}$ that outputs $\texttt{TRUE}$ if, and only if, $X$ is a
final set (i.e., $X \cap F \neq \emptyset$). In the implementation, this is a
flag that the determinization module sets when creating $X$.

We will also use set-lists, denoted by $\cstateslists$ in the algorithm, which
are linked lists of state-sets. For this data structure, we assume a method
$\cstateslists.\texttt{clear}$ to empty the list,
$\cstateslists.\texttt{add}(X)$ to add $X$ at the end, and
$\cstateslists.\texttt{swap}(\cstateslists')$ to swap the content between two
lists. To iterate over each element $X$ in the list, we conveniently write
``$\textbf{for all} \ X \in \cstateslists$''. We assume any straightforward
implementation of set-list that takes constant time for each call to these
methods. Finally, during the algorithm, we use two set-list variables,
$\cstateslists$ and $\cstateslists'$. We assume that $\cstateslists$,
$\cstateslists'$, $\ds$, and $\cdet$ can be globally accessed by all methods.

\section{Main algorithm}
The main method of Algorithm~\ref{alg:evaluation} is $\textproc{Evaluate}$,
which receives as input an $\EVA$ $\cE = (Q,\delta,q_0,F)$ and the document $b_0
\ldots b_{n}$. Recall that $b_0 \ldots b_{n-1}$ is the original document and
$b_n = \osymbol$ is the \texttt{EOF} symbol. The algorithm starts by
initializing the determinization module $\cdet$ with $\cE$ at line~$2$. The
\texttt{initialize} method is for storing $\cE$ inside $\cdet$ and using it
later during the determinization. Then we do the initialization of
$\cstateslists$ and $\cstateslists'$ by calling $\textproc{InitializeLists}$
(line 3). This method clears both lists (lines 15-16) and adds the state-set
$X_0$ to $\cstateslists$ (line 20). This state-set represents $\{q_0\}$, namely,
the determinization initial set. For this, $\cdet$ provides a method
$\cdet.\texttt{initialStateSet}$ that outputs $X_0 = \{q_0\}$ (line 17). Then we
initiate $X_0.\citer$ with $-1$ (line 18) and fill $X_0.\coid$ with the empty
output node (line 19). Intuitively, no iterations have accessed $X_0$ yet, and
the $\cE^{\odet}$-run at $X_0$ only has the empty output.

The evaluation algorithm processes the document $b_0 \ldots b_{n}$ symbol by
symbol, from $i=0$ to $n$ (line 4). During the $i$-th iteration, $\cstateslists$
keeps all state-sets $X$ that can be reached by runs reading $b_0\ldots
b_{i-1}$, and $X.\coid$ is the node storing all outputs of these runs. Instead,
$\cstateslists'$ will contain the new state-sets after reading the new letter
$b_i$. Intuitively, to build $\cstateslists'$ from $\cstateslists$ we fire all
state-sets $X$ in $\cstateslists$ one-by-one (line 5) by calling the method
$\cdet.\texttt{next}(X,b_i)$ with symbol $b_i$ (line 6). As explained
previously, this method gives a list $\ell$ of pairs $(S, X')$, where there is a
transition $(X, b_i, S, X') \in \delta^{\odet}$. Then, if $(S, X') \in \ell$, we
must extend the output sequences in $X.\coid$ with $(S, i)$, and store them in
$X'$. We do this updating by calling the method
$\textproc{UpdateSets}(X,\ell,i)$ (line 8), discussed below. If the list $\ell$
is empty (e.g., $\cdet$ detects that by reading $b_i$ from $X$, there is no way
to continue), then we call the node manager $\ds$ and mark $X.\coid$ for the
garbage collection (line 10). Notice that, by using the determinization manager,
we can detect the right place to reuse memory in the algorithm. Finally, when we
end firing all state-sets in $\cstateslists$, we swap the two lists and clean
$\cstateslists'$ to start the iteration again (lines 11-12).

\section{Update method}
The workhorse of the evaluation algorithm is $\textproc{UpdateSets}(X, \ell,i)$,
which is in charge of updating each state-set $X'$ by extending the outputs in
$X.\coid$ with $(S,i)$, for each $(S, X') \in \!\ell$. For this purpose, we
iterate over each $(S, X') \in \ell$, create a copy $\coid'$ of $X.\coid$, and
extend all output sequences in $\coid'$ with $(S,i)$ if $S\neq \emptyset$ (lines
28-30). Next, we need to update $X'.\coid$ depending on whether it is the first
time or not that we reach $X'$. For this, we use variable $X'.\citer$: if
$X'.\citer < i$, then we update $X'.\citer$ to $i$, add $X'$ to
$\cstateslists'$, and set $X'.\coid$ equal to $\coid'$ (lines 31-34). Otherwise,
it is not the first time that we reach $X'$, and we need to union the output
sequences in $X'.\coid$ with the ones in $\coid'$ (lines~35-36).

For the correctness of $\textproc{UpdateSets}$, we cannot reach any state-set
$X$ in two consecutive iterations (i.e., both in the $(i-1)$- and $i$-th
iteration for some $i$). Otherwise, we could use $X.\coid$ by reading and
updating it simultaneously, possibly erasing its content. For example, suppose
that in the $(i-1)$-th iteration of the algorithm there are two set-lists
holding different set-states, $X_1$ and $X_2$. If in the $i$-th iteration, $X_1$
transitions into $X_2$, and $X_2$ transitions into another state $X'$, then
$X_2.n$ would be read and written in the same step, which can be dangerous if
the writing happens before the reading.   For this reason, during the rewriting
module we duplicate the logical $\VA$, alternating between even and odd
positions. This construction forces that, at this stage, we cannot reach any
state-set $X$ in two consecutive iterations as needed.


$\textproc{UpdateSets}$ passes and updates the outputs from the $(i-1)$- to the
$i$-th layer, which is the heaviest part of the algorithm. Therefore, it is
crucial to perform $\textproc{UpdateSets}$ as efficiently as possible. For this,
we use linked-lists and $\citer$ variables, which allows us to check in a single
instruction whether $X'$ is already in $\cstateslists'$ or not. These tricks are
standard in the implementation of RegEx engines~\citep{cox2007regular}. The
novelty is then to incorporate these optimizations in a new evaluation algorithm
for computing all matches.

\begin{figure}
	\centering
	\begin{tikzpicture}[every text node part/.style={align=center},
    lt/.style={outer sep=0mm, inner sep=2mm, node distance=8mm, text height=2mm, text centered},
    st/.style={outer sep=0mm, inner sep=0.5mm, node distance=8mm, text height=2mm, text centered},
    ed/.style={->},
    nc/.style={outer sep=0mm, inner sep=2mm, node distance=8.5mm, text height=2mm, text centered,text=black!50}]
    
    %\draw[step=.5cm, dashed, draw=black!20] (-1,-6) grid (7,1);
            
    \node[lt] (l0) at (0,0) {\Large \texttt{t}\vphantom{y}};
    \node[lt,right of=l0] (l1) {\Large \texttt{h}\vphantom{y}};
    \node[lt,right of=l1] (l2) {\Large \texttt{a}\vphantom{y}};
    \node[lt,right of=l2] (l3) {\Large \texttt{t}\vphantom{y}};
    \node[lt,right of=l3] (l4) {\Large \texttt{h}\vphantom{y}};
    \node[lt,right of=l4] (l5) {\Large \texttt{a}\vphantom{y}};
    \node[lt,right of=l5] (l6) {\Large \texttt{t}\vphantom{y}};
    \node[lt,right of=l6] (l7) {\Large \texttt{h}\vphantom{y}};
    \node[lt,right of=l7] (l8) {\Large \texttt{a}\vphantom{y}};
    \node[lt,right of=l8] (l9) {\Large \texttt{t}\vphantom{y}};
    \node[lt,right of=l9] (l10) {\Large $\osymbol$ \vphantom{y}};
    
    \draw (-0.5, -0.15) to (8.8, -0.15);
    
    \node[st] (s0) at (-0.325,-0.45) {$0$};
    \node[st, right of=s0] (s1) {$1$};
    \node[st, right of=s1] (s2) {$2$};
    \node[st, right of=s2] (s3) {$3$};
    \node[st, right of=s3] (s4) {$4$};
    \node[st, right of=s4] (s5) {$5$};
    \node[st, right of=s5] (s6) {$5$};
    \node[st, right of=s6] (s7) {$5$};
    \node[st, right of=s7] (s8) {$5$};
    \node[st, right of=s8] (s9) {$5$};
    \node[st, right of=s9] (s10) {$5$};
    \node[st, right of=s10] (s11) {$5$};
    {
        \footnotesize
        \node[nc] at ($(s0.south east)+(0.05,0)$) {$n_0$};
        \node[nc] at ($(s1.south east)+(0.05,0)$) {$n_1$};
        \node[nc] at ($(s2.south east)+(0.05,0)$) {$n_1$};
        \node[nc] at ($(s3.south east)+(0.05,0)$) {$n_1$};
        \node[nc] at ($(s4.south east)+(0.05,0)$) {$n_3$};
        \node[nc] at ($(s5.south east)+(0.05,0)$) {$n_3$};
        \node[nc] at ($(s6.south east)+(0.05,0)$) {$n_3$};
        \node[nc] at ($(s7.south east)+(0.05,0)$) {$n_3$};
        \node[nc] at ($(s8.south east)+(0.05,0)$) {$n_6$};
        \node[nc] at ($(s9.south east)+(0.05,0)$) {$n_6$};
        \node[nc] at ($(s10.south east)+(0.05,0)$) {$n_6$};
        \node[nc] at ($(s11.south east)+(0.05,0)$) {$n_9$};
    }
    
    \node[st, below of=s1, node distance=6mm] (t1) {$0$};
    \node[st, right of=t1] (t2) {$0$};
    \node[st, right of=t2] (t3) {$0$};
    \node[st, right of=t3] (t4) {$1$};
    \node[st, right of=t4] (t5) {$2$};
    \node[st, right of=t5] (t6) {$3$};
    \node[st, right of=t6] (t7) {$4$};
    
    {
        \footnotesize
        \node[nc] at ($(t1.south east)+(0.05,0)$) {$n_0$};
        \node[nc] at ($(t2.south east)+(0.05,0)$) {$n_0$};
        \node[nc] at ($(t3.south east)+(0.05,0)$) {$n_0$};
        \node[nc] at ($(t4.south east)+(0.05,0)$) {$n_2$};
        \node[nc] at ($(t5.south east)+(0.05,0)$) {$n_2$};
        \node[nc] at ($(t6.south east)+(0.05,0)$) {$n_2$};
        \node[nc] at ($(t7.south east)+(0.05,0)$) {$n_5$};
    }
    
    \node[st, below of=t4, node distance=6mm] (u4) {$0$};
    \node[st, right of=u4] (u5) {$0$};
    \node[st, right of=u5] (u6) {$0$};
    \node[st, right of=u6] (u7) {$1$};
    \node[st, below of=s8, node distance=6mm] (u8) {$2$};
    \node[st, right of=u8] (u9) {$3$};
    \node[st, right of=u9] (u10) {$4$};
    
    {
        \footnotesize
        \node[nc] at ($(u4.south east)+(0.05,0)$) {$n_0$};
        \node[nc] at ($(u5.south east)+(0.05,0)$) {$n_0$};
        \node[nc] at ($(u6.south east)+(0.05,0)$) {$n_0$};
        \node[nc] at ($(u7.south east)+(0.05,0)$) {$n_4$};
        \node[nc] at ($(u8.south east)+(0.05,0)$) {$n_4$};
        \node[nc] at ($(u9.south east)+(0.05,0)$) {$n_4$};
        \node[nc] at ($(u10.south east)+(0.05,0)$) {$n_7$};
    }
    
    \node[st, below of=u7, node distance=6mm] (v7) {$0$};
    \node[st, below of=u8, node distance=6mm] (v8) {$0$};
    \node[st, right of=v8] (v9) {$0$};
    \node[st, right of=v9] (v10) {$1$};
    
    {
        \footnotesize
        \node[nc] at ($(v7.south east)+(0.05,0)$) {$n_0$};
        \node[nc] at ($(v8.south east)+(0.05,0)$) {$n_0$};
        \node[nc] at ($(v9.south east)+(0.05,0)$) {$n_0$};
        \node[nc] at ($(v10.south east)+(0.05,0)$) {$n_8$};
    }
    
    \node[st, below of=v10, node distance=6mm] (w10) {$0$};
    \node[st, below of=s11, node distance=6mm] (w11) {$0$};
    
    {
        \footnotesize
        \node[nc] at ($(w10.south east)+(0.05,0)$) {$n_0$};
        \node[nc] at ($(w11.south east)+(0.05,0)$) {$n_0$};
    }
    
    \tikzstyle{every state}=[draw=black,text=black,inner sep=0pt, minimum size=9mm]
    
    \node [state] (n0) at (0.1,-5.5) {$\bot$};
    
    \node [state] (n1) at ($(n0)+(1.8,-2)$) {$[x,\!0$};
    \node [state] (n2) at ($(n0)+(1.8,0)$) {$[x,\!3$};
    \node [state] (n3) at ($(n0)+(1.8,2)$) {$[x,\!6$};
    
    \node [state] (n9) at ($(n0)+(0,2)$) {$[x,\!9$};
    
    \node [state] (n4) at ($(n2)+(1.8,-2)$) {$x\rangle,\!4$};
    \node [state] (n5) at ($(n2)+(1.8,0)$) {$x\rangle,\!7$};
    \node [state] (n6) at ($(n2)+(1.8,2)$) {$x\rangle,\!10$};
    
    \node [state] (n7) at ($(n4)+(1.8,0)$) {$\cup$};
    
    \node [state] (n8) at ($(n7)+(1.8,2)$) {$\cup$};
    
    {
        \small
        \node[nc] at ($(n0.north west)+(-0.2,0.2)$) {$n_0\!:$};
        \node[nc] at ($(n1)+(-0.8,0)$) {$n_1\!:$};
        \node[nc] at ($(n2.north west)+(-0.2,0.2)$) {$n_2\!:$};
        \node[nc] at ($(n3.north west)+(-0.2,0.2)$) {$n_4\!:$};
        \node[nc] at ($(n4.north west)+(-0.2,0.2)$) {$n_3\!:$};
        \node[nc] at ($(n5.north west)+(-0.2,0.2)$) {$n_5\!:$};
        \node[nc] at ($(n6.north west)+(-0.2,0.2)$) {$n_7\!:$};
        \node[nc] at ($(n7)+(0,0.7)$) {$n_6\!:$};
        \node[nc] at ($(n8.north west)+(-0.2,0.2)$) {$n_9\!:$};
        \node[nc] at ($(n9.north west)+(-0.2,0.2)$) {$n_8\!:$};
    }
    
    \draw[<-] (n0) to (n1);
    \draw[<-] (n0) to (n2);
    \draw[<-] (n0) to (n3);
    
    \draw[<-] (n1) to (n4);
    \draw[<-] (n2) to (n5);
    \draw[<-] (n3) to (n6);
    
    \draw[<-] (n4) to (n7);
    \draw[<-] (n5) to (n7);
    
    \draw[<-] (n7) to (n8);
    \draw[<-] (n6) to (n8);
    \draw[<-] (n0) to (n9);
    
    \draw[decorate,decoration={brace, amplitude=5, mirror}] (9,-2.6)  -- node[right=2mm,rotate=90,anchor=north] {\cstateslistss} (9,-0.2) ;
    
    \draw[decorate,decoration={brace, amplitude=5, mirror}] (9,-8)  -- node[right=2mm,rotate=90,anchor=north] {Enumerable compact set} (9,-2.8) ;
\end{tikzpicture}
	\caption{An example of how Algorithm~\ref{alg:evaluation} works.}
	\label{fig:algo-run}
\end{figure}

\begin{example}
	Figure~\ref{fig:algo-run} displays a graphic of how
	Algorithm~\ref{alg:evaluation} would run with eVA $\cA_0$ from
	Example~\ref{ex:lva-eva} and document $d_0$. Below document $d_0$, we draw
	as columns the \cstateslistss{} that are computed after reading the
	corresponding symbol. For this example, each state-set is a single state,
	and then each number in a \cstateslists{} corresponds to a state in $\cA_0$.
	Below each state-set $X$, we draw in grey the node $X.\coid$ from the ECS
	that the algorithm constructs.
\end{example}

\section{Next index (optimization)}
A significant step for the evaluation algorithm is the call to the
$\cdet.\texttt{next}$ function (line 6). Given a state-set $X$ and a symbol
$b_i$, the first time that the algorithm calls $\cdet.\texttt{next}(X, b_i)$ the
$\cdet$ module must compute a list with all pairs $(S, X')$ such that it holds
that $(X,a,S,X') \in \delta^\odet$, and save it in its cache. Then, for later
calls to  $\cdet.\texttt{next}(X, b_i)$, the $\cdet$ module must quickly find
this list in the cache. The evaluation uses the $\cdet$'s cache multiple times,
making it  one of the heaviest parts of the computation. To decrease the load of
the algorithm, we add an index to each state set $X$, which quickly allows
finding the next state-set given a $b_i$. Currently, \rematch\ only supports
ASCII documents; therefore, we implement this index as an array with 128
entries. This array on each state-set allows us to quickly find the next
state-set, considerably improving the performance of the evaluation algorithm.
Here, for the next index, we are sacrificing space versus time. Of course, a
more compact next index (e.g., for non-ASCII documents) could save space during
the evaluation. We leave this for future work.

\section{Enumeration}
When we get to the end of the document, $\cstateslists$ contains all state-sets
$X$ that can be reached by runs of $\cE^\odet$ when reading $b_0 \ldots b_n$. In
particular, $X.\coid$ has the ECS node that encodes all outputs of these runs.
Then we call $\textproc{Enumerate}$ (line 13), which iterates over all $X \in
\cstateslists$ that are final, and enumerates the output sequences in $X.\coid$
by calling the enumerate method of the node manager (lines 23-25).
%
As described in Chapter~\ref{chpt:output}, we can perform an early output
enumeration whenever we reach a final state at the $i$-th iteration. For ease of
presentation, we code Algorithm~\ref{alg:evaluation} with the enumeration
procedure after reading the whole document. %Instead, in REmatch's
implementation, we call the enumeration method whenever we reach a final state.

