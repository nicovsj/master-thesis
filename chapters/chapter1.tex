% !TeX root = ../main.tex !TeX spellcheck = en_US

This section introduces REQL, a RegEx Query Language for information extraction,
that we implement in \rematch. The language is an extension of the classical
RegEx syntax (e.g. POSIX Basic Regular Expressions) familiar to most users. On
the other hand, the semantics is inspired by the document spanner
framework~\citep{FaginKRV15} that captures all appearances of a pattern in the
document. 

In the following, we present the formal syntax and semantics of REQL, and
provide several examples of REQL queries.

\section{Documents and spans}

We follow the theoretical framework of documents and spans introduced
in~\citep{FaginKRV15}. For us, a document $d$ is simply a string over some
finite alphabet (e.g. the ASCII charset, UTF-8, or a similar encoding
scheme)\footnote{Note that a multi-line document is simply a single string.}. We
write $d = a_0a_1 \ldots a_{n-1}$ to denote a document of length $|d| = n$ where
$a_{i}$ is the $i$th symbol (note that the first symbol starts from
$0$)\footnote{In \citet{FaginKRV15}, the first position is $1$. We use $0$ to be
compliant with programming languages and RegEx engines that use $0$ as the start
position.}. An example of a document is given in Figure~\ref{fig-doc}. A
\emph{span} of a document $d$ (also called a \emph{match}) is a pair $s =
[i,j\rangle$ of natural numbers $i$ and $j$ with $0 \leq i \leq j\leq |d|$. In
that case, $s$ is associated with the contiguous region of the document~$d$
whose content is the substring of $d$ from position $i$ to position $j-1$. We
denote this substring by $d(s)$ or $d([i,j\rangle)$. For instance,
$d_1([0,4\rangle) = \texttt{that}$, since this is the content of the string $d$
in positions 0 through~3.

Notice that if $i = j$, then $d(s) = d(i,j) = \varepsilon$, the empty string.
Given two spans $s_1 = [i_1, j_1\rangle$ and $s_2 = [i_2, j_2\rangle$ such that
$j_1 = i_2$, we define their \emph{concatenation} as $s_1 \cdot s_2=[i_1,
j_2\rangle$. The set of all spans of $d$ is denoted by $\sub(d)$.


\section{Syntax} 
Syntactically, REQL is similar to standard regular expressions, apart from a
special construct \texttt{!x\{e\}}, which states that a substring matching
\texttt{e} should be stored into the variable name \texttt{x}. Formally, the
syntax of REQL queries can be defined as follows:
$$
\begin{array}{rcl}
\texttt{e} & \coloneqq & \texttt{a} \ \mid \ \texttt{.} \ \mid \
\texttt{[w]} \ \mid \ \texttt{[\textasciicircum w]} \ \mid \
%\ \texttt{\textasciicircum} \ \mid \ \texttt{\$}  \\
\texttt{!x\{e\}}  \ \mid \\
& &  \texttt{e} \texttt{e} \ \mid \
\texttt{e|e} \ \mid \ \texttt{e*} \ \mid \ \texttt{e+} \ \mid \ \texttt{e?} \ \mid \ \texttt{e\{n,m\}}
\end{array}
$$
Here, $\texttt{a}$ is a character (e.g., ASCII charset or UTF-8), the dot symbol
is a wildcard for any character, and $\texttt{[w]}$ or
$\texttt{[\textasciicircum w]}$ are a char class or the negation of a char
class, respectively, where $\texttt{w}$ declares a set of characters. We use the
standard notation of ranges of ASCII characters found in POSIX for declaring
char classes (e.g. $\texttt{[a-z]}$, $\texttt{[A-Z0-9apt]}$, etc.) and write
$\text{set}(\texttt{w})$ to denote the set of characters represented by
$\texttt{w}$ (e.g. $\text{set}(\texttt{a-z}) = \{a, b, ..., z\}$). Moreover,
$\texttt{x}$ is a variable name where the character \texttt{!} is used to
differentiate a variable name from a letter or string of the alphabet. This,
along with the use of~\texttt{\{} and~\texttt{\}} for delimiting the captured
subregex is the only special notation where we differ from POSIX. Finally,
$\texttt{n}$ and $\texttt{m}$ are numbers such that $0 \leq \texttt{n} \leq
\texttt{m}$. In the REmatch system, REQL also allows the usual regex
abbreviations for character classes (e.g., \texttt{$\backslash$d} for a digit,
or \texttt{$\backslash$w} for a word, etc), and the escaping of special syntax
characters with a backslash to treat them as a normal character (e.g.,
\texttt{$\backslash$!}, \texttt{$\backslash$}+, etc.). However, we do not
include them in the formal definition in order to keep the presentation
concise\footnote{We remark that the start-of-line symbol
($\texttt{\textasciicircum}$) and end-of-line symbol ($\texttt{\$}$) are
currently not supported in \rematch. However, adding them is a straightforwar d
exercise.}.


\begin{figure}
	\vspace*{-7mm}
	\begin{center}
		%\hspace*{-0.7cm}
		\resizebox{.4\linewidth}{!}{
			\begin{tikzpicture}[baseline=0ex, node distance=1.5cm, every text
				node part/.style={align=center}] \spanNode{0}{t}
				\spanNodeP{0}{1}{h} \spanNodeP{1}{2}{a} \spanNodeP{2}{3}{t}
				\spanNodeP{3}{4}{h} \spanNodeP{4}{5}{a} \spanNodeP{5}{6}{t}
				\spanNodeP{6}{7}{h} \spanNodeP{7}{8}{a} \spanNodeP{8}{9}{t}
				\draw ($(l0) + (-0.1, -0.2)$) to ($(l9) + (0.1, -0.2)$); \node
				at ($(l0) + (-0.6, 0.05)$) {$d_1 :=\ $};
			\end{tikzpicture}
		}
	\end{center}
	\vspace*{-7mm}
	\caption{A sample document for illustration purposes.}
	\label{fig-doc}
\end{figure}



\begin{example}\label{example:basic} To give a preliminary example of how REQL
works, assume that we would like to extract all the occurrences of the word
``that" from a text document. This can be done in REQL as follows:
$$
\texttt{e0} := \texttt{\textbf{!x\{}that\textbf{\}}}
$$
Intuitively, the query captures the positions of a substring \texttt{that} into
the variable $x$. This query also illustrates a key feature of our semantics
(defined below): there can be overlapping matches. To make this more clear,
consider the document $d_1$ in Figure \ref{fig-doc}. The query above will result
in precisely three matches for the variable \texttt{x}, corresponding to the
three occurrences of the substring \texttt{that} in the document we are
processing. The first match will be in positions $[0,4\rangle$, the second in
$[3,7\rangle$, and the last match in $[6,10\rangle$. We notice that the middle
match $[3,7\rangle$ will not be captured by most regular expression tools,
unless some sort of a look-around operator is used.
\end{example}

The reader could notice that the above syntax is so general that one can define
the capture of the same variable multiple times. For instance, a query like
$\texttt{!x\{a!x\{b\}\}}$ defines the capture of $\texttt{x}$ twice. For this
reason, REQL has some simple syntactic restrictions to use variables correctly.
Let $\Var(\texttt{e})$ be the set of all variables names used in $\texttt{e}$.
We say that a REQL query is \emph{well-designed}\footnote{In~\citet{FaginKRV15},
expressions satisfying these conditions are called \emph{functional}.} if every
subquery~$\texttt{e}$ satisfies the following four conditions: (1)~if
$\texttt{e}=\texttt{!x\{e1\}}$, then $\texttt{x} \notin \Var(\texttt{e1})$,
(2)~if $\texttt{e} = \texttt{e1}\,\texttt{e2}$, then
$\Var(\texttt{e1})\cap\Var(\texttt{e2}) = \emptyset$; (3)~if $\texttt{e} =
\texttt{e1}|\texttt{e2}$, then $\Var(\texttt{e1})=\Var(\texttt{e2})$; and (4)~if
$\texttt{e}$ is equal to \texttt{e1*}, \texttt{e1+}, \texttt{e1?} or
\texttt{e1\{n,m\}}, then $\Var(\texttt{e1}) = \emptyset$. One can easily check
that queries $\texttt{!x\{a!x\{b\}\}}$, $\texttt{!x\{a\}!x\{b\}}$,
$\texttt{a|!x\{b\}}$, or $\texttt{(!x\{a\}b)*}$ are not well-designed. Instead,
queries like $\texttt{!x\{a\}!y\{b\}}$, $\texttt{!x\{a\}|!x\{b\}}$, or
$\texttt{!x\{a\}(b)*}$ do satisfy all conditions and then are well-designed.
Note that, as shown in~\citet{FaginKRV15}, the well-designed condition does not
diminish the query language's expressive power. Then, from now on, we will
consider all the queries we evaluate to be well-designed.

\section{Semantics} 
We define the matches extracted by REQL in terms of mappings. Formally, a
\emph{mapping} for a document $d$ is a (partial) function $\mu$ from variables
to spans of~$d$. Intuitively, a mapping represents a single match that a REQL
query makes on a document $d$. For instance, in our previous example, the query
$\texttt{e0}$ will produce three mappings as its output: $\mu_1$, with $\mu_1(x)
= [0,4\rangle$, $\mu_2$, with $\mu_2(x) = [3,7\rangle$, and $\mu_3$, with
$\mu_3(x) = [6,10\rangle$. We write $\dom(\mu)$ to denote the domain of $\mu$
and $\mu_1 \cup \mu_2$ for the disjoint union of mappings whenever $\dom(\mu_1)
\cap \dom(\mu_2) = \emptyset$. We also use the notation $[x \to s]$ to define
the singleton mapping that only maps $x$ to the span $s$ (e.g., $\mu_1 = [x \to
[0,4\rangle]$), and use $\emptyset$ for the trivial empty mapping (where the
domain is the empty set). 

\begin{table}
	\begin{align*}
		\semd{\texttt{e}} & \ = \  \{ \mu \mid (s, \mu) \in \sempd{\texttt{e}} \}\\
		%  \sempd{\varepsilon} & \ = \  \{ (s, \emptyset) \mid s \in \sub(d) \text{ and } d(s)=\varepsilon\}\\
		\sempd{\texttt{a}} & \ = \  \{ (s, \emptyset) \mid s \in \sub(d) \text{ and } d(s) = \texttt{a} \}\\
		\sempd{\texttt{.}} & \ = \  \{ ([i,i+1\rangle, \emptyset) \mid 0 \leq i < |d| \}\\
		\sempd{\texttt{[w]}} & \ = \  \{ (s, \emptyset) \mid s \in \sub(d) \text{ and } d(s) \in  \text{set}(\texttt{w}) \}\\
		\sempd{\texttt{[\textasciicircum w]}} & \ = \  \{ (s, \emptyset) \mid s \in \sub(d) \text{ and } d(s) \notin  \text{set}(\texttt{w}) \}\\
		\sempd{\texttt{!x\{e\}}} & \ = \  \{ (s, \mu) \mid \exists (s, \mu') \in \sempd{\texttt{e}}: \ d(s) \neq \epsilon \text{ and }
		\mu =  \mu'  \cup [x \to s] \ \} \\
		\sempd{\texttt{e1} \, \texttt{e2}} & \ = \  \{ (s, \mu) \mid
		\exists (s_1, \mu_1) \in \sempd{\texttt{e1}} \ \exists (s_2, \mu_2) \in \sempd{\texttt{e2}}: \\
		& \;\;\;\;\;\;\;\;\;\;\;\;\;\;\;\; s = s_1 \cdot s_2 \text{ and } \mu = \mu_1 \cup \mu_2 \}\\
		\sempd{\texttt{e1|e2}} & \ = \  \sempd{\texttt{e1}} \cup \sempd{\texttt{e2}}\\
		\sempd{\texttt{e*}} & \ = \  {\textstyle\bigcup}_{h\geq 0}\sempd{\texttt{e}^h} \quad \text{where } \sempd{\texttt{e}^0} = \{([i,i\rangle, \emptyset) \mid 0 \leq i \leq |d|\}\ \\
		& \qquad\qquad\qquad\qquad \text{ and } \sempd{\texttt{e}^{h+1}} = \sempd{\texttt{e}(\texttt{e}^h)} \text{ for } h \geq 0 \\
		\sempd{\texttt{e+}} & \ = \  \sempd{\texttt{e}\, \texttt{(e*)}} \\
		\sempd{\texttt{e?}} & \ = \  \sempd{\texttt{e}^0} \cup \sempd{\texttt{e}} \\
		\sempd{\texttt{e\{n,m\}}} & \ = \  \sempd{\texttt{e}^n(\texttt{e?})^{m-n}}
	\end{align*}
	\vspace*{-10pt}
	\caption{The inductive semantics of REQL queries.}
	\label{tab-semantics}	
	\vspace{-5mm}
\end{table}

With the formalism of mappings, we can give a concise declarative semantics for
REQL, similarly as in \citet{MaturanaRV17}. This is done in
Table~\ref{tab-semantics}. The semantics is defined by structural induction on
$\texttt{e}$ and has two layers. The first layer, $\sempd{\texttt{e}}$, defines
the set of all pairs $(s,\mu)$ with $s \in \sub(d)$ and $\mu$ a mapping such
that (1) $\texttt{e}$ successfully matches the substring $d(s)$ and (2) $\mu$
results as a consequence of this successful match. For example, the REQL query
$a$ matches all substrings of input document $d$ equal to $a$, but results in
only the empty mapping. On the other hand, $\texttt{!x\{e\}}$ matches all
substrings that are matched by $\texttt{e}$, but assigns to $x$ the
\emph{non-empty} span $s$ that delimits the substring being matched, while
preserving the previous variable assignments. Similarly, in the case of
concatenation $\texttt{e1}\,\texttt{e2}$ we join the mapping defined on the left
with the one defined on the right. Notice that these mappings will not share any
variables, given that the expression is assumed to be well-formed. The second
layer, $\semd{\texttt{e}}$, then simply gives us the mappings that $\texttt{e}$
defines when matching the entire document. Note that when $\texttt{e}$ is an
ordinary regular expression (i.e., no variables), then the empty mapping is
output if the entire documentccontains a match of $\texttt{e}$, and no mapping
is output otherwise. 

In the following, we provide several examples from English text analysis to
grasp the power of REQL for information extraction and to see its differences
concerning classical RegEx. The reader can test these examples and other REQL
queries in our \rematch{} beta demo available at \url{www.rematch.cl}.
\begin{example}\label{example:ex1} A typical task in language analysis is
detecting words with particular roots, or more precisely lexemes, which are
basic units of meaning. For example, one could be interested in words in the
English language that start with the prefix `a'. To extract all such words from
a text, we can use the following REQL expression\footnote{This example is for
illustration purposes. The actual expression would allow arbitrary spacing and
sentence punctuation, and allow matching the first word in the sentence.}:
$$
\texttt{e1} = \Vtextvisiblespace \texttt{!word\{[Aa]}\backslash \texttt{w+\}[}\Vtextvisiblespace \texttt{.]}
$$
\noindent
where $\Vtextvisiblespace$ denotes a single white space, and
$\backslash$\texttt{w} denotes the char class of words characters, as commonly
used is Perl-compatible regular expressions. Note that in
$\texttt{[}\Vtextvisiblespace \texttt{.]}$ the \texttt{.} denotes the dot symbol
and not a wildcard. This is consistent with the classic RegEx syntax, since a
wildcard symbol is useless for defining a char class.

In \texttt{e1} we are looking for a word staring with the letter 'a'. To assure
we will capture the entire word, we preceded it by a space, and we required that
after reading it we see either a space or a dot symbol. If we evaluate
$\texttt{e1}$ over document $d_2$ in Figure~\ref{fig-doc2}, we will get four
mappings assigning the variable \texttt{word} to the spans $\Span{4}{7}$,
$\Span{11}{13}$, $\Span{14}{21}$, and $\Span{22}{31}$, representing the words
``ant'', ``an'', ``amazing'', and ``architect'', respectively.
\end{example}
In classical RegEx, round parentheses denote a capture group for extracting a
substring. That is,  \texttt{(R)} will extract what is matched by the RegEx
\texttt{R}. We could therefore try to express \texttt{e1} from
Example~\ref{example:ex1} by the expression: $\Vtextvisiblespace
\texttt{([Aa]}\backslash \texttt{w+)[}\Vtextvisiblespace \texttt{.]}$ which
replaces REQL's capture variables with a capture group. However, when evaluated
over the document $d_2$ in Figure~\ref{fig-doc2}, one fewer output will be
produced; namely: the span $\Span{14}{21}$ corresponding to the word ``amazing''
will be missing. This is due to leftmost-longest match semantics deployed by
classic RegEx engines, which will consume the white space following the word
``an'', therefore preventing the expression from matching ``amazing''. A typical
workaround for this problem is the use of \emph{look-ahead operators}, which
allow to check whether a string is present starting from some position. A RegEx
expression equivalent to \texttt{e1} would then be $\Vtextvisiblespace
\texttt{([Aa]}\backslash \texttt{w+)(?=[}\Vtextvisiblespace \texttt{.])}$ which
upon matching a word will look-ahead for a space or a dot, without advancing
with the current match. In general, using look-ahead operators is somewhat
cumbersome, and, as we show below, is not sufficient to capture all the matches
in some cases. In contrast, REQL supports the all-match semantics by default: it
returns a match for \emph{every} span in the document where the specified
pattern occurs.


\begin{figure}
	\begin{center}
	\resizebox{\linewidth}{!}{
			\begin{tikzpicture}[baseline=0ex, node distance=.1cm, every text
				node part/.style={align=center}] \spanNode{0}{T}
				\spanNodeP{0}{1}{h} \spanNodeP{1}{2}{e} \spanNodeP{2}{3}{ }
				\spanNodeP{3}{4}{a} \spanNodeP{4}{5}{n} \spanNodeP{5}{6}{t}
				\spanNodeP{6}{7}{ } \spanNodeP{7}{8}{i} \spanNodeP{8}{9}{s}
				\spanNodeP{9}{10}{ } \spanNodeP{10}{11}{a} \spanNodeP{11}{12}{n}
				\spanNodeP{12}{13}{ } \spanNodeP{13}{14}{a}
				\spanNodeP{14}{15}{m} \spanNodeP{15}{16}{a}
				\spanNodeP{16}{17}{z} \spanNodeP{17}{18}{i}
				\spanNodeP{18}{19}{n} \spanNodeP{19}{20}{g} \spanNodeP{20}{21}{
				} \spanNodeP{21}{22}{a} \spanNodeP{22}{23}{r}
				\spanNodeP{23}{24}{c} \spanNodeP{24}{25}{h}
				\spanNodeP{25}{26}{i} \spanNodeP{26}{27}{t}
				\spanNodeP{27}{28}{e} \spanNodeP{28}{29}{c}
				\spanNodeP{29}{30}{t} \spanNodeP{30}{31}{.}
	
				\draw ($(l0) + (-0.1, -0.2)$) to ($(l31) + (0.1, -0.2)$); \node
				at ($(l0) + (-0.6, 0.05)$) {$d_2 :=\ $};
				
			\end{tikzpicture}
		}
	\end{center}
	\vspace*{-7mm}
	\caption{Document containing a sentence from the book ``What is a man?'' by Mark Twain.}
	\label{fig-doc2}
\end{figure}

\begin{example} \label{example:ex2} Suppose that the user wants to process the
	English text into $k$-grams (i.e., $k$ consecutive words in a text) that
	satisfy some particular pattern. Specifically, suppose this user wants to
	extract all $2$-grams where each word begins with the letter `a'. We can
	extract them by running the following REQL query:
	$$
	\texttt{e2} := \Vtextvisiblespace \texttt{\textbf{!w1\{}[Aa]}\backslash \texttt{w+\textbf{\}}}\Vtextvisiblespace \texttt{\textbf{!w2\{}[Aa]}\backslash \texttt{w+\textbf{\}}[}\Vtextvisiblespace \texttt{.]}
	$$
	Note that $\texttt{e2}$ is the extension of $\texttt{e1}$ where now we use
	two variable names, called $\texttt{w1}$ and $\texttt{w2}$, for obtaining
	the substrings of the first and second words, respectively. For instance, if
	we run $\texttt{e2}$ over $d_2$ in Figure~\ref{fig-doc2} we will get
	mappings:
	$$
	\begin{array}{cc}
		[\texttt{w1} \mapsto \Span{11}{13}, \texttt{w2} \mapsto \Span{14}{21}] & 
		{[}\texttt{w1} \mapsto \Span{14}{21}, \texttt{w2} \mapsto \Span{22}{31}] \vspace{1mm}\\
	\end{array}
	$$
	representing the $2$-grams ``an amazing'' and ``amazing architect''.%,
	respectively.
\end{example}
Note that the previous query cannot be obtained by any RegEx engine without
``look-arounds'', given that $2$-grams can overlap. %, as in the last example.
\begin{example}\label{example:ex3} We end by showing another capacity of REQL
	for extracting contextual information, another feature not supported by
	RegEx. Suppose that, in addition to the $2$-grams, the user wants to extract
	the sentence where the match happens. This additional information could be
	useful for understanding the context where these $2$-grams are used. For
	this, we can modify our query $\texttt{e2}$ as follows:
$$
\texttt{e3}  :=  \backslash \texttt{.}\texttt{\textbf{!sent\{}} \ \texttt{[\textasciicircum .]*} \Vtextvisiblespace \texttt{\textbf{!w1\{}[Aa]}\backslash \texttt{w+\textbf{\}}}\Vtextvisiblespace \texttt{\textbf{!w2\{}[Aa]}\backslash \texttt{w+\textbf{\}}}  \texttt{(}\Vtextvisiblespace\texttt{[\textasciicircum .]*)?}\backslash .  \ \texttt{\textbf{\}}}
$$
	Here, the new variable \texttt{sent} will store the information containing
	the sentence where the $2$-gram occurs. The reader can check that if we
	evaluate \texttt{e3} over $d_2$, then we will obtain the mappings of
	Example~\ref{example:ex2} where each mapping will have in addition the
	variable $\texttt{sent}$ maps to $\Span{0}{31}$, which represents the whole
	sentence. Interestingly, this semantics context of a match cannot be
	extracted by RegEx, even if we use look-ahead operators. The main issue for
	look-ahead operators is that due to the leftmost-longest semantics, no two
	matches starting at the same position can be returned, which is not an issue
	in our case.
\end{example}

\section{Comparison with RegEx and Document Spanners} 
As previously explained, we base REQL on RegEx syntax and the semantics of the
Document Spanners framework. The purpose of reusing RegEx syntax is that users
feel familiar with the query language and operators. However, as the previous
examples show, the all-match semantics differs from the leftmost-longest
semantics from classical RegEx. Therefore, we introduce REQL as a new query
language rather than presenting it as an extension of RegEx. 

The class of regular expressions with extraction variables was first introduced
by~\citet{FaginKRV15}. Their framework is called Document Spanners, and it
formalizes the process of extracting relations from text documents. We base REQL
semantics on Document Spanners, although there are several differences. First,
while Document Spanners are a theoretical tool for information extraction, REQL
is a user-oriented query language with a programming syntax based on RegEx.
Second, the semantics proposed in Document Spanners is anchored at the beginning
and end of the document (like regular expressions in theory), where REQL is
unanchored; namely, the query is evaluated anywhere in the document (similar to
RegEx engines). Third, REQL semantics disallows capturing $\epsilon$ substrings
(see $\sempd{\texttt{!x\{e\}}}$ in Table~\ref{tab-semantics}), where Document
Spanners allow this. %Although this
difference is minimal, it implies that Document Spanners' language is slightly
more expressive. We decided to remove $\epsilon$-substring capturing since it is
not very helpful for users, and its removal simplifies several optimization
procedures in \rematch.

Comparing \rematch\ with classical RegEx engines, they both use classical
operators and shortcuts, and  match a substring from any position in the
document, as opposed to the theoretical approaches to regular expressions. The
main difference comes from capture variables and the all-match semantics. As we
have seen, the combination of the two prevent simulating REQL's capture
variables in RegEx using capture groups. Similarly, all-match semantics lies
outside of the scope of RegEx, since matches starting at the same position
cannot be simulated even using the look-around operators. As a minimal example
for this, consider the document $d = \texttt{aaa}$, and the REQL expression
$\texttt{e = !\{a*\}}$, which will produce six matches in this case, each one
corresponding to a non-empty substring of~$d$. On the other hand, using
look-ahead will not help us to capture this in RegEx. The most obvious way would
be to use an expression of the form $\texttt{(?=(a*))}$, however, at position 0,
only the match $[\texttt{w1} \mapsto \Span{0}{3}]$ will be produced, and, for
example $[\texttt{w1} \mapsto \Span{0}{1}]$ will be omitted, since they both
start at the same position.

