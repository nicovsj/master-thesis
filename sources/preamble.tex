%%%%%%%%%
%   Packages	 %
%%%%%%%%%

% Floats
\usepackage{graphicx}
\usepackage{float}
% \floatstyle{boxed}
\restylefloat{figure}
\usepackage{color}
\usepackage{caption}
\usepackage{subcaption}
\usepackage{varwidth}
\usepackage[multiple]{footmisc}

% Math packages
\usepackage{amsmath}
\usepackage{amsfonts}
\usepackage{amssymb}
\usepackage{stmaryrd}
\usepackage{mathtools}
\usepackage{algorithm}
\usepackage[noend]{algpseudocode}

% Closest font to Times New Roman
\usepackage{mathptmx}

% To make pretty tables
\usepackage{booktabs}
\usepackage{multirow}

% To avoid underfull errors in the bibliography
\usepackage{etoolbox}
\apptocmd{\sloppy}{\hbadness 10000\relax}{}{}

% To make cites and references
\usepackage[hidelinks,pdfusetitle,pdfdisplaydoctitle,hyperfootnotes=false]{hyperref}
\usepackage[natbibapa]{apacite} 
\usepackage{doi}
\renewcommand{\doitext}{}

% TiKZ
\usepackage{tikz}
\usetikzlibrary{automata,graphs,positioning,fit,backgrounds,chains,arrows,decorations.pathmorphing,decorations.pathreplacing,calc}

\input{sources/ecs.tikzstyles}

%--------- NEW ENVIRONMENTS --------- You are free to remove or use it
\newtheorem{definition}{\bf Definition}[chapter]
\newtheorem{property}{Property}[chapter]
\newtheorem{claim}{Claim}[chapter]
\newtheorem{lemma}{\bf Lemma}[chapter]
\newtheorem{proposition}{Proposition}[chapter]
\newtheorem{theorem}{\noindent \bf Theorem}[chapter]
\newtheorem{corollary}{\bf Corollary}[chapter]
\newtheorem{pf}{Proof}[chapter]
\newtheorem{example}{\bf Example}[chapter]
\newtheorem{remark}{Remark}[chapter]

% Todo notes

\usepackage{todonotes}
%UNCOMMENT THIS TO REMOVE COMMENTS
%\usepackage[disable]{todonotes}

\newcommand{\cristian}[1]{\todo[inline, color=red!20]{{\bf C:} #1}}

% En caso de que el título sea muy largo, se puede ajustar el espacio
% antes y después de este en las dos primeras páginas para que quede centrado.

%\makeatletter
%  \setlength{\beftitle}{105\p@\@plus24\p@}
%  \setlength{\afttitle}{65\p@}
%\makeatother